%*******************************************************
% Abstract
%*******************************************************
%\renewcommand{\abstractname}{Abstract}
\pdfbookmark[1]{Abstract}{Abstract}
\begingroup
\let\clearpage\relax
\let\cleardoublepage\relax
\let\cleardoublepage\relax

\chapter*{Abstract}
\draftnote{
    This document is examining the ratio of the parts which can be automatically optimized by Polly.
    Further the reasons why larger parts are rejected and the theoretical impact if they would not be rejected.
}

\begin{comment}
    \begin{itemize}
        \item Plakative Stichworte
        \item Kalauer?
        \item Ausblick
        \item Ergebnis
    \end{itemize}
\end{comment}
\draftnote{
    \begin{itemize}
        \item Background: Give a brief introducing notice about the motivation for conducting the study.
        \item Objective: Describe the aim of the study, including the object under examination, the focus, and the perspective.
        \item Method: Describe which research method was used to examine the object (e.g., experimental design, number and kind of participants, selection criteria, data collection and analysis procedures).
        \item Results: Describe the main findings.
        \item Limitations: Describe the major limitations of the research, if any.
        \item Conclusion: Describe the impact of the results.
        \item lengths:
            \begin{itemize}
                \item background: one sentence
                \item important elements: objective, method, result and conclusion
            \end{itemize}
        \item keywords
    \end{itemize}
}

\vfill

\begin{otherlanguage}{ngerman}
    \pdfbookmark[1]{Zusammenfassung}{Zusammenfassung}
    \chapter*{Zusammenfassung}
    \draftnote{
    Dieses Dokument beschäftigt sich mit dem Anteil der durch Polly automatisch optimierbaren Bereiche.
    Darüber hinaus wird der Frage nachgegangen, mit welchen Gründen größere Bereiche abgelehnt werden und auch mit den theoritischen Auswirkungen, wenn diese erkannt worden wären.
    }
\end{otherlanguage}

\endgroup

\vfill
