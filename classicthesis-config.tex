%Added by Stefan Huber
\usepackage{comment}
\usepackage[dvipsnames]{xcolor}
\usepackage{minted} % should be added before scrhack %newfloat now working
\usepackage{ltablex}
\usepackage{wrapfig}
\usepackage{smartdiagram}
\usesmartdiagramlibrary{additions}

% ****************************************************************************************************
% classicthesis-config.tex
% formerly known as loadpackages.sty, classicthesis-ldpkg.sty, and classicthesis-preamble.sty
% Use it at the beginning of your ClassicThesis.tex, or as a LaTeX Preamble
% in your ClassicThesis.{tex,lyx} with %Added by Stefan Huber
\usepackage{comment}
\usepackage[dvipsnames]{xcolor}
\usepackage{minted} % should be added before scrhack %newfloat now working
\usepackage{wrapfig}
\usepackage{smartdiagram}
\usesmartdiagramlibrary{additions}

% ****************************************************************************************************
% classicthesis-config.tex
% formerly known as loadpackages.sty, classicthesis-ldpkg.sty, and classicthesis-preamble.sty
% Use it at the beginning of your ClassicThesis.tex, or as a LaTeX Preamble
% in your ClassicThesis.{tex,lyx} with %Added by Stefan Huber
\usepackage{comment}
\usepackage[dvipsnames]{xcolor}
\usepackage{minted} % should be added before scrhack %newfloat now working
\usepackage{wrapfig}
\usepackage{smartdiagram}
\usesmartdiagramlibrary{additions}

% ****************************************************************************************************
% classicthesis-config.tex
% formerly known as loadpackages.sty, classicthesis-ldpkg.sty, and classicthesis-preamble.sty
% Use it at the beginning of your ClassicThesis.tex, or as a LaTeX Preamble
% in your ClassicThesis.{tex,lyx} with %Added by Stefan Huber
\usepackage{comment}
\usepackage[dvipsnames]{xcolor}
\usepackage{minted} % should be added before scrhack %newfloat now working
\usepackage{wrapfig}
\usepackage{smartdiagram}
\usesmartdiagramlibrary{additions}

% ****************************************************************************************************
% classicthesis-config.tex
% formerly known as loadpackages.sty, classicthesis-ldpkg.sty, and classicthesis-preamble.sty
% Use it at the beginning of your ClassicThesis.tex, or as a LaTeX Preamble
% in your ClassicThesis.{tex,lyx} with \input{classicthesis-config}
% ****************************************************************************************************
% If you like the classicthesis, then I would appreciate a postcard.
% My address can be found in the file ClassicThesis.pdf. A collection
% of the postcards I received so far is available online at
% http://postcards.miede.de
% ****************************************************************************************************


% ****************************************************************************************************
% 0. Set the encoding of your files. UTF-8 is the only sensible encoding nowadays. If you can't read
% äöüßáéçèê∂åëæƒÏ€ then change the encoding setting in your editor, not the line below. If your editor
% does not support utf8 use another editor!
% ****************************************************************************************************
\PassOptionsToPackage{utf8}{inputenc}
\usepackage{inputenc}

% ****************************************************************************************************
% 1. Configure classicthesis for your needs here, e.g., remove "drafting" below
% in order to deactivate the time-stamp on the pages
% ****************************************************************************************************
\PassOptionsToPackage{eulerchapternumbers,listings,%drafting,%
        pdfspacing,%floatperchapter,%linedheaders,%
        dottedtoc,
        subfig,beramono,eulermath,parts}{classicthesis}
% ********************************************************************
% Available options for classicthesis.sty
% (see ClassicThesis.pdf for more information):
% drafting
% parts nochapters linedheaders
% eulerchapternumbers beramono eulermath pdfspacing minionprospacing
% tocaligned dottedtoc manychapters
% listings floatperchapter subfig
% ********************************************************************


% ****************************************************************************************************
% 2. Personal data and user ad-hoc commands
% ****************************************************************************************************
\newcommand{\myTitle}{A Classic Thesis Style\xspace}
\newcommand{\mySubtitle}{An Homage to The Elements of Typographic Style\xspace}
\newcommand{\myDegree}{Nahe am Bachelor\xspace}
\newcommand{\myName}{Stefan Huber\xspace}
\newcommand{\myProf}{Put name here\xspace}
\newcommand{\myOtherProf}{Put name here\xspace}
\newcommand{\mySupervisor}{Andreas Simbürger\xspace}
\newcommand{\myFaculty}{Fakultät für Informatik und Mathematik\xspace}
\newcommand{\myDepartment}{Department\xspace}
\newcommand{\myUni}{Universität Passau\xspace}
\newcommand{\myLocation}{Passau\xspace}
\newcommand{\myTime}{September 2017\xspace}
\newcommand{\myVersion}{version 1\xspace}

% ********************************************************************
% Setup, finetuning, and useful commands
% ********************************************************************
\newcounter{dummy} % necessary for correct hyperlinks (to index, bib, etc.)
\newlength{\abcd} % for ab..z string length calculation
\providecommand{\mLyX}{L\kern-.1667em\lower.25em\hbox{Y}\kern-.125emX\@}
\newcommand{\eg}{e.\,g.}
\newcommand{\Eg}{E.\,g.}
% ****************************************************************************************************


% ****************************************************************************************************
% 3. Loading some handy packages
% ****************************************************************************************************
% ********************************************************************
% Packages with options that might require adjustments
% ********************************************************************
\PassOptionsToPackage{british}{babel}   % change this to your language(s)
% Spanish languages need extra options in order to work with this template
%\PassOptionsToPackage{spanish,es-lcroman}{babel}
\usepackage{babel}

\usepackage{csquotes}
\PassOptionsToPackage{%
    %backend=biber, %instead of bibtex
    backend=bibtex8,bibencoding=ascii,%
    language=auto,%
    style=numeric-comp,%
    %style=authoryear-comp, % Author 1999, 2010
    %bibstyle=authoryear,dashed=false, % dashed: substitute rep. author with ---
    sorting=nyt, % name, year, title
    maxbibnames=10, % default: 3, et al.
    %backref=true,%
    natbib=true % natbib compatibility mode (\citep and \citet still work)
}{biblatex}
    \usepackage{biblatex}

\PassOptionsToPackage{fleqn}{amsmath}       % math environments and more by the AMS
    \usepackage{amsmath}

% ********************************************************************
% General useful packages
% ********************************************************************
\PassOptionsToPackage{T1}{fontenc} % T2A for cyrillics
    \usepackage{fontenc}
\usepackage{textcomp} % fix warning with missing font shapes
\usepackage{scrhack} % fix warnings when using KOMA with listings package
\usepackage{xspace} % to get the spacing after macros right
\usepackage{mparhack} % get marginpar right
\usepackage{fixltx2e} % fixes some LaTeX stuff --> since 2015 in the LaTeX kernel (see below)
%\usepackage[latest]{latexrelease} % will be used once available in more distributions (ISSUE #107)
\PassOptionsToPackage{printonlyused,smaller}{acronym}
    \usepackage{acronym} % nice macros for handling all acronyms in the thesis
    %\renewcommand{\bflabel}[1]{{#1}\hfill} % fix the list of acronyms --> no longer working
    %\renewcommand*{\acsfont}[1]{\textsc{#1}}
    \renewcommand*{\aclabelfont}[1]{\acsfont{#1}}
% ****************************************************************************************************


% ****************************************************************************************************
% 4. Setup floats: tables, (sub)figures, and captions
% ****************************************************************************************************
\usepackage{tabularx} % better tables
    \setlength{\extrarowheight}{3pt} % increase table row height
\newcommand{\tableheadline}[1]{\multicolumn{1}{c}{\spacedlowsmallcaps{#1}}}
\newcommand{\myfloatalign}{\centering} % to be used with each float for alignment
\usepackage{caption}
% Thanks to cgnieder and Claus Lahiri
% http://tex.stackexchange.com/questions/69349/spacedlowsmallcaps-in-caption-label
% [REMOVED DUE TO OTHER PROBLEMS, SEE ISSUE #82]
%\DeclareCaptionLabelFormat{smallcaps}{\bothIfFirst{#1}{~}\MakeTextLowercase{\textsc{#2}}}
%\captionsetup{font=small,labelformat=smallcaps} % format=hang,
\captionsetup{font=small} % format=hang,
\usepackage{subfig}
% ****************************************************************************************************


% ****************************************************************************************************
% 5. Setup code listings
% ****************************************************************************************************
\begin{comment}
    \usepackage{listings}
    %\lstset{emph={trueIndex,root},emphstyle=\color{BlueViolet}}%\underbar} % for special keywords
    \lstset{language=C++,
        morekeywords={PassOptionsToPackage,selectlanguage},
        keywordstyle=\color{NavyBlue},%\bfseries,
        basicstyle=\small\ttfamily,
        %identifierstyle=\color{NavyBlue},
        commentstyle=\color{Green}\ttfamily,
        stringstyle=\rmfamily,
        numbers=left,
        numberstyle=\scriptsize,%\tiny
        %stepnumber=5,
        numbersep=8pt,
        showstringspaces=false,
        breaklines=true,
        %frameround=ftff,
        %frame=single,
        belowcaptionskip=.75\baselineskip,
        frame=L
    }
\end{comment}
% ****************************************************************************************************


% ****************************************************************************************************
% 6. PDFLaTeX, hyperreferences and citation backreferences
% ****************************************************************************************************
% ********************************************************************
% Using PDFLaTeX
% ********************************************************************
\PassOptionsToPackage{pdftex,hyperfootnotes=false,pdfpagelabels}{hyperref}
    \usepackage{hyperref}  % backref linktocpage pagebackref
\pdfcompresslevel=9
\pdfadjustspacing=1
\PassOptionsToPackage{pdftex}{graphicx}
    \usepackage{graphicx}


% ********************************************************************
% Hyperreferences
% ********************************************************************
\hypersetup{%
    %draft, % = no hyperlinking at all (useful in b/w printouts)
    colorlinks=true, linktocpage=true, pdfstartpage=3, pdfstartview=FitV,%
    % uncomment the following line if you want to have black links (e.g., for printing)
    %colorlinks=false, linktocpage=false, pdfstartpage=3, pdfstartview=FitV, pdfborder={0 0 0},%
    breaklinks=true, pdfpagemode=UseNone, pageanchor=true, pdfpagemode=UseOutlines,%
    plainpages=false, bookmarksnumbered, bookmarksopen=true, bookmarksopenlevel=1,%
    hypertexnames=true, pdfhighlight=/O,%nesting=true,%frenchlinks,%
    urlcolor=webbrown, linkcolor=RoyalBlue, citecolor=webgreen, %pagecolor=RoyalBlue,%
    %urlcolor=Black, linkcolor=Black, citecolor=Black, %pagecolor=Black,%
    pdftitle={\myTitle},%
    pdfauthor={\textcopyright\ \myName, \myUni, \myFaculty},%
    pdfsubject={},%
    pdfkeywords={},%
    pdfcreator={pdfLaTeX},%
    pdfproducer={LaTeX with hyperref and classicthesis}%
}

% ********************************************************************
% Setup autoreferences
% ********************************************************************
% There are some issues regarding autorefnames
% http://www.ureader.de/msg/136221647.aspx
% http://www.tex.ac.uk/cgi-bin/texfaq2html?label=latexwords
% you have to redefine the makros for the
% language you use, e.g., american, ngerman
% (as chosen when loading babel/AtBeginDocument)
% ********************************************************************
\makeatletter
\@ifpackageloaded{babel}%
    {%
        \addto\extrasamerican{%
            \renewcommand*{\figureautorefname}{Figure}%
            \renewcommand*{\tableautorefname}{Table}%
            \renewcommand*{\partautorefname}{Part}%
            \renewcommand*{\chapterautorefname}{Chapter}%
            \renewcommand*{\sectionautorefname}{Section}%
            \renewcommand*{\subsectionautorefname}{Section}%
            \renewcommand*{\subsubsectionautorefname}{Section}%
        }%
        \addto\extrasbritish{%
            \renewcommand*{\figureautorefname}{Figure}%
            \renewcommand*{\tableautorefname}{Table}%
            \renewcommand*{\partautorefname}{Part}%
            \renewcommand*{\chapterautorefname}{Chapter}%
            \renewcommand*{\sectionautorefname}{Section}%
            \renewcommand*{\subsectionautorefname}{Section}%
            \renewcommand*{\subsubsectionautorefname}{Section}%
        }%
        \addto\extrasngerman{%
            \renewcommand*{\paragraphautorefname}{Absatz}%
            \renewcommand*{\subparagraphautorefname}{Unterabsatz}%
            \renewcommand*{\footnoteautorefname}{Fu\"snote}%
            \renewcommand*{\FancyVerbLineautorefname}{Zeile}%
            \renewcommand*{\theoremautorefname}{Theorem}%
            \renewcommand*{\appendixautorefname}{Anhang}%
            \renewcommand*{\equationautorefname}{Gleichung}%
            \renewcommand*{\itemautorefname}{Punkt}%
        }%
        % Fix to getting autorefs for subfigures right (thanks to Belinda Vogt for changing the definition)
        \providecommand{\subfigureautorefname}{\figureautorefname}%
    }{\relax}
\makeatother


% ****************************************************************************************************
% 7. Last calls before the bar closes
% ****************************************************************************************************
% ********************************************************************
% Development Stuff
% ********************************************************************
\listfiles
%\PassOptionsToPackage{l2tabu,orthodox,abort}{nag}
%   \usepackage{nag}
%\PassOptionsToPackage{warning, all}{onlyamsmath}
%   \usepackage{onlyamsmath}

% ********************************************************************
% Last, but not least...
% ********************************************************************
\usepackage{classicthesis}
% ****************************************************************************************************


% ****************************************************************************************************
% 8. Further adjustments (experimental)
% ****************************************************************************************************
% ********************************************************************
% Changing the text area
% ********************************************************************
%\linespread{1.05} % a bit more for Palatino
%\areaset[current]{312pt}{761pt} % 686 (factor 2.2) + 33 head + 42 head \the\footskip
%\setlength{\marginparwidth}{7em}%
%\setlength{\marginparsep}{2em}%

% ********************************************************************
% Using different fonts
% ********************************************************************
%\usepackage[oldstylenums]{kpfonts} % oldstyle notextcomp
%\usepackage[osf]{libertine}
%\usepackage[light,condensed,math]{iwona}
%\renewcommand{\sfdefault}{iwona}
%\usepackage{lmodern} % <-- no osf support :-(
%\usepackage{cfr-lm} %
%\usepackage[urw-garamond]{mathdesign} <-- no osf support :-(
%\usepackage[default,osfigures]{opensans} % scale=0.95
%\usepackage[sfdefault]{FiraSans}
% ****************************************************************************************************


%Customization Stefan Huber
%custom colors
\definecolor{defaultColor}{HTML}{1E90FF}
\definecolor{llvmIrColor}{HTML}{228B22}
\definecolor{nonLlvmIrColor}{HTML}{B22222}

%minted configure
\setminted{
    linenos,
    breaklines
}
\providecommand*{\listingautorefname}{Listing}
\newenvironment{code}{\captionsetup{type=listing}}{}
%\SetupFloatingEnvironment{listing}{name=Program code} %newfloat package
%\renewcommand{\listoflistingscaption}{List of Program Code} %float package

%Tikz
\usepackage{tikz}
\usetikzlibrary{arrows.meta, positioning, shapes, snakes, fit}
\tikzset{
    node distance = 5mm and 15mm,
    defaultNode/.style = {rectangle, rounded corners, draw=gray, drop shadow, very thick, bottom color=defaultColor!50!white, text centered, top color=white, minimum height=1.2em, minimum width=3em},
    llvmIrNode/.style = {defaultNode, bottom color=llvmIrColor!50!white},
    nonLlvmIrNode/.style = {defaultNode, bottom color=nonLlvmIrColor!50!white},
    legendNode/.style = {rectangle},
    defaultPath/.style = {-{Stealth[length=4mm]}, color=defaultColor, line width=0.2em, draw, rounded corners},
    llvmIrPath/.style = {defaultPath, color=llvmIrColor},
    nonLlvmIrPath/.style = {defaultPath, color=nonLlvmIrColor}
}
\newcommand{\legend}[0]{
    \begin{tikzpicture}
        \coordinate (firstRowCenter);
        \coordinate[left=of firstRowCenter] (firstRowLeft);
        \coordinate[right=of firstRowCenter] (firstRowRight);
        \path[llvmIrPath] (firstRowLeft) to (firstRowRight);
        \node[legendNode, right=of firstRowRight] {Passing LLVM-IR};

        \node[llvmIrNode, below=of firstRowCenter] (secondRowCenter) {Node};
        \coordinate[left=of secondRowCenter] (secondRowLeft);
        \coordinate[right=of secondRowCenter] (secondRowRight);
        \node[legendNode, right=of secondRowRight] {Node using LLVM-IR};

        \coordinate[below=of secondRowCenter] (thirdRowCenter);
        \coordinate[left=of thirdRowCenter] (thirdRowLeft);
        \coordinate[right=of thirdRowCenter] (thirdRowRight);
        \path[nonLlvmIrPath] (thirdRowLeft) to (thirdRowRight);
        \node[legendNode, right=of thirdRowRight] {Passing non-LLVM-IR};

        \node[nonLlvmIrNode, below=of thirdRowCenter] (fourthRowCenter) {Node};
        \coordinate[left=of fourthRowCenter] (fourthRowLeft);
        \coordinate[right=of fourthRowCenter] (fourthRowRight);
        \node[legendNode, right=of fourthRowRight] {Node not using LLVM-IR};
    \end{tikzpicture}
}

%Shortcuts
\newcommand{\cfg}{\ac{CFG} }
\newcommand{\scop}{\ac{SCoP} }
\newcommand{\scops}{\ac{SCoP}s }
\newcommand{\llvmir}{\ac{LLVM-IR} }
\newcommand{\llvm}{\ac{LLVM} }
\newcommand{\opt}{LLVM Optimizer }
\newcommand{\generator}{LLVM Code Generator }
\newcommand{\linker}{LLVM Linker }

%temporary commands while drafting
\newcommand{\draftnote}[1]{\textit{\textbf{\textcolor{red}{#1}}}}

% ****************************************************************************************************
% If you like the classicthesis, then I would appreciate a postcard.
% My address can be found in the file ClassicThesis.pdf. A collection
% of the postcards I received so far is available online at
% http://postcards.miede.de
% ****************************************************************************************************


% ****************************************************************************************************
% 0. Set the encoding of your files. UTF-8 is the only sensible encoding nowadays. If you can't read
% äöüßáéçèê∂åëæƒÏ€ then change the encoding setting in your editor, not the line below. If your editor
% does not support utf8 use another editor!
% ****************************************************************************************************
\PassOptionsToPackage{utf8}{inputenc}
\usepackage{inputenc}

% ****************************************************************************************************
% 1. Configure classicthesis for your needs here, e.g., remove "drafting" below
% in order to deactivate the time-stamp on the pages
% ****************************************************************************************************
\PassOptionsToPackage{eulerchapternumbers,listings,%drafting,%
        pdfspacing,%floatperchapter,%linedheaders,%
        dottedtoc,
        subfig,beramono,eulermath,parts}{classicthesis}
% ********************************************************************
% Available options for classicthesis.sty
% (see ClassicThesis.pdf for more information):
% drafting
% parts nochapters linedheaders
% eulerchapternumbers beramono eulermath pdfspacing minionprospacing
% tocaligned dottedtoc manychapters
% listings floatperchapter subfig
% ********************************************************************


% ****************************************************************************************************
% 2. Personal data and user ad-hoc commands
% ****************************************************************************************************
\newcommand{\myTitle}{A Classic Thesis Style\xspace}
\newcommand{\mySubtitle}{An Homage to The Elements of Typographic Style\xspace}
\newcommand{\myDegree}{Nahe am Bachelor\xspace}
\newcommand{\myName}{Stefan Huber\xspace}
\newcommand{\myProf}{Put name here\xspace}
\newcommand{\myOtherProf}{Put name here\xspace}
\newcommand{\mySupervisor}{Andreas Simbürger\xspace}
\newcommand{\myFaculty}{Fakultät für Informatik und Mathematik\xspace}
\newcommand{\myDepartment}{Department\xspace}
\newcommand{\myUni}{Universität Passau\xspace}
\newcommand{\myLocation}{Passau\xspace}
\newcommand{\myTime}{September 2017\xspace}
\newcommand{\myVersion}{version 1\xspace}

% ********************************************************************
% Setup, finetuning, and useful commands
% ********************************************************************
\newcounter{dummy} % necessary for correct hyperlinks (to index, bib, etc.)
\newlength{\abcd} % for ab..z string length calculation
\providecommand{\mLyX}{L\kern-.1667em\lower.25em\hbox{Y}\kern-.125emX\@}
\newcommand{\eg}{e.\,g.}
\newcommand{\Eg}{E.\,g.}
% ****************************************************************************************************


% ****************************************************************************************************
% 3. Loading some handy packages
% ****************************************************************************************************
% ********************************************************************
% Packages with options that might require adjustments
% ********************************************************************
\PassOptionsToPackage{british}{babel}   % change this to your language(s)
% Spanish languages need extra options in order to work with this template
%\PassOptionsToPackage{spanish,es-lcroman}{babel}
\usepackage{babel}

\usepackage{csquotes}
\PassOptionsToPackage{%
    %backend=biber, %instead of bibtex
    backend=bibtex8,bibencoding=ascii,%
    language=auto,%
    style=numeric-comp,%
    %style=authoryear-comp, % Author 1999, 2010
    %bibstyle=authoryear,dashed=false, % dashed: substitute rep. author with ---
    sorting=nyt, % name, year, title
    maxbibnames=10, % default: 3, et al.
    %backref=true,%
    natbib=true % natbib compatibility mode (\citep and \citet still work)
}{biblatex}
    \usepackage{biblatex}

\PassOptionsToPackage{fleqn}{amsmath}       % math environments and more by the AMS
    \usepackage{amsmath}

% ********************************************************************
% General useful packages
% ********************************************************************
\PassOptionsToPackage{T1}{fontenc} % T2A for cyrillics
    \usepackage{fontenc}
\usepackage{textcomp} % fix warning with missing font shapes
\usepackage{scrhack} % fix warnings when using KOMA with listings package
\usepackage{xspace} % to get the spacing after macros right
\usepackage{mparhack} % get marginpar right
\usepackage{fixltx2e} % fixes some LaTeX stuff --> since 2015 in the LaTeX kernel (see below)
%\usepackage[latest]{latexrelease} % will be used once available in more distributions (ISSUE #107)
\PassOptionsToPackage{printonlyused,smaller}{acronym}
    \usepackage{acronym} % nice macros for handling all acronyms in the thesis
    %\renewcommand{\bflabel}[1]{{#1}\hfill} % fix the list of acronyms --> no longer working
    %\renewcommand*{\acsfont}[1]{\textsc{#1}}
    \renewcommand*{\aclabelfont}[1]{\acsfont{#1}}
% ****************************************************************************************************


% ****************************************************************************************************
% 4. Setup floats: tables, (sub)figures, and captions
% ****************************************************************************************************
\usepackage{tabularx} % better tables
    \setlength{\extrarowheight}{3pt} % increase table row height
\newcommand{\tableheadline}[1]{\multicolumn{1}{c}{\spacedlowsmallcaps{#1}}}
\newcommand{\myfloatalign}{\centering} % to be used with each float for alignment
\usepackage{caption}
% Thanks to cgnieder and Claus Lahiri
% http://tex.stackexchange.com/questions/69349/spacedlowsmallcaps-in-caption-label
% [REMOVED DUE TO OTHER PROBLEMS, SEE ISSUE #82]
%\DeclareCaptionLabelFormat{smallcaps}{\bothIfFirst{#1}{~}\MakeTextLowercase{\textsc{#2}}}
%\captionsetup{font=small,labelformat=smallcaps} % format=hang,
\captionsetup{font=small} % format=hang,
\usepackage{subfig}
% ****************************************************************************************************


% ****************************************************************************************************
% 5. Setup code listings
% ****************************************************************************************************
\begin{comment}
    \usepackage{listings}
    %\lstset{emph={trueIndex,root},emphstyle=\color{BlueViolet}}%\underbar} % for special keywords
    \lstset{language=C++,
        morekeywords={PassOptionsToPackage,selectlanguage},
        keywordstyle=\color{NavyBlue},%\bfseries,
        basicstyle=\small\ttfamily,
        %identifierstyle=\color{NavyBlue},
        commentstyle=\color{Green}\ttfamily,
        stringstyle=\rmfamily,
        numbers=left,
        numberstyle=\scriptsize,%\tiny
        %stepnumber=5,
        numbersep=8pt,
        showstringspaces=false,
        breaklines=true,
        %frameround=ftff,
        %frame=single,
        belowcaptionskip=.75\baselineskip,
        frame=L
    }
\end{comment}
% ****************************************************************************************************


% ****************************************************************************************************
% 6. PDFLaTeX, hyperreferences and citation backreferences
% ****************************************************************************************************
% ********************************************************************
% Using PDFLaTeX
% ********************************************************************
\PassOptionsToPackage{pdftex,hyperfootnotes=false,pdfpagelabels}{hyperref}
    \usepackage{hyperref}  % backref linktocpage pagebackref
\pdfcompresslevel=9
\pdfadjustspacing=1
\PassOptionsToPackage{pdftex}{graphicx}
    \usepackage{graphicx}


% ********************************************************************
% Hyperreferences
% ********************************************************************
\hypersetup{%
    %draft, % = no hyperlinking at all (useful in b/w printouts)
    colorlinks=true, linktocpage=true, pdfstartpage=3, pdfstartview=FitV,%
    % uncomment the following line if you want to have black links (e.g., for printing)
    %colorlinks=false, linktocpage=false, pdfstartpage=3, pdfstartview=FitV, pdfborder={0 0 0},%
    breaklinks=true, pdfpagemode=UseNone, pageanchor=true, pdfpagemode=UseOutlines,%
    plainpages=false, bookmarksnumbered, bookmarksopen=true, bookmarksopenlevel=1,%
    hypertexnames=true, pdfhighlight=/O,%nesting=true,%frenchlinks,%
    urlcolor=webbrown, linkcolor=RoyalBlue, citecolor=webgreen, %pagecolor=RoyalBlue,%
    %urlcolor=Black, linkcolor=Black, citecolor=Black, %pagecolor=Black,%
    pdftitle={\myTitle},%
    pdfauthor={\textcopyright\ \myName, \myUni, \myFaculty},%
    pdfsubject={},%
    pdfkeywords={},%
    pdfcreator={pdfLaTeX},%
    pdfproducer={LaTeX with hyperref and classicthesis}%
}

% ********************************************************************
% Setup autoreferences
% ********************************************************************
% There are some issues regarding autorefnames
% http://www.ureader.de/msg/136221647.aspx
% http://www.tex.ac.uk/cgi-bin/texfaq2html?label=latexwords
% you have to redefine the makros for the
% language you use, e.g., american, ngerman
% (as chosen when loading babel/AtBeginDocument)
% ********************************************************************
\makeatletter
\@ifpackageloaded{babel}%
    {%
        \addto\extrasamerican{%
            \renewcommand*{\figureautorefname}{Figure}%
            \renewcommand*{\tableautorefname}{Table}%
            \renewcommand*{\partautorefname}{Part}%
            \renewcommand*{\chapterautorefname}{Chapter}%
            \renewcommand*{\sectionautorefname}{Section}%
            \renewcommand*{\subsectionautorefname}{Section}%
            \renewcommand*{\subsubsectionautorefname}{Section}%
        }%
        \addto\extrasbritish{%
            \renewcommand*{\figureautorefname}{Figure}%
            \renewcommand*{\tableautorefname}{Table}%
            \renewcommand*{\partautorefname}{Part}%
            \renewcommand*{\chapterautorefname}{Chapter}%
            \renewcommand*{\sectionautorefname}{Section}%
            \renewcommand*{\subsectionautorefname}{Section}%
            \renewcommand*{\subsubsectionautorefname}{Section}%
        }%
        \addto\extrasngerman{%
            \renewcommand*{\paragraphautorefname}{Absatz}%
            \renewcommand*{\subparagraphautorefname}{Unterabsatz}%
            \renewcommand*{\footnoteautorefname}{Fu\"snote}%
            \renewcommand*{\FancyVerbLineautorefname}{Zeile}%
            \renewcommand*{\theoremautorefname}{Theorem}%
            \renewcommand*{\appendixautorefname}{Anhang}%
            \renewcommand*{\equationautorefname}{Gleichung}%
            \renewcommand*{\itemautorefname}{Punkt}%
        }%
        % Fix to getting autorefs for subfigures right (thanks to Belinda Vogt for changing the definition)
        \providecommand{\subfigureautorefname}{\figureautorefname}%
    }{\relax}
\makeatother


% ****************************************************************************************************
% 7. Last calls before the bar closes
% ****************************************************************************************************
% ********************************************************************
% Development Stuff
% ********************************************************************
\listfiles
%\PassOptionsToPackage{l2tabu,orthodox,abort}{nag}
%   \usepackage{nag}
%\PassOptionsToPackage{warning, all}{onlyamsmath}
%   \usepackage{onlyamsmath}

% ********************************************************************
% Last, but not least...
% ********************************************************************
\usepackage{classicthesis}
% ****************************************************************************************************


% ****************************************************************************************************
% 8. Further adjustments (experimental)
% ****************************************************************************************************
% ********************************************************************
% Changing the text area
% ********************************************************************
%\linespread{1.05} % a bit more for Palatino
%\areaset[current]{312pt}{761pt} % 686 (factor 2.2) + 33 head + 42 head \the\footskip
%\setlength{\marginparwidth}{7em}%
%\setlength{\marginparsep}{2em}%

% ********************************************************************
% Using different fonts
% ********************************************************************
%\usepackage[oldstylenums]{kpfonts} % oldstyle notextcomp
%\usepackage[osf]{libertine}
%\usepackage[light,condensed,math]{iwona}
%\renewcommand{\sfdefault}{iwona}
%\usepackage{lmodern} % <-- no osf support :-(
%\usepackage{cfr-lm} %
%\usepackage[urw-garamond]{mathdesign} <-- no osf support :-(
%\usepackage[default,osfigures]{opensans} % scale=0.95
%\usepackage[sfdefault]{FiraSans}
% ****************************************************************************************************


%Customization Stefan Huber
%custom colors
\definecolor{defaultColor}{HTML}{1E90FF}
\definecolor{llvmIrColor}{HTML}{228B22}
\definecolor{nonLlvmIrColor}{HTML}{B22222}

%minted configure
\setminted{
    linenos,
    breaklines
}
\providecommand*{\listingautorefname}{Listing}
\newenvironment{code}{\captionsetup{type=listing}}{}
%\SetupFloatingEnvironment{listing}{name=Program code} %newfloat package
%\renewcommand{\listoflistingscaption}{List of Program Code} %float package

%Tikz
\usepackage{tikz}
\usetikzlibrary{arrows.meta, positioning, shapes, snakes, fit}
\tikzset{
    node distance = 5mm and 15mm,
    defaultNode/.style = {rectangle, rounded corners, draw=gray, drop shadow, very thick, bottom color=defaultColor!50!white, text centered, top color=white, minimum height=1.2em, minimum width=3em},
    llvmIrNode/.style = {defaultNode, bottom color=llvmIrColor!50!white},
    nonLlvmIrNode/.style = {defaultNode, bottom color=nonLlvmIrColor!50!white},
    legendNode/.style = {rectangle},
    defaultPath/.style = {-{Stealth[length=4mm]}, color=defaultColor, line width=0.2em, draw, rounded corners},
    llvmIrPath/.style = {defaultPath, color=llvmIrColor},
    nonLlvmIrPath/.style = {defaultPath, color=nonLlvmIrColor}
}
\newcommand{\legend}[0]{
    \begin{tikzpicture}
        \coordinate (firstRowCenter);
        \coordinate[left=of firstRowCenter] (firstRowLeft);
        \coordinate[right=of firstRowCenter] (firstRowRight);
        \path[llvmIrPath] (firstRowLeft) to (firstRowRight);
        \node[legendNode, right=of firstRowRight] {Passing LLVM-IR};

        \node[llvmIrNode, below=of firstRowCenter] (secondRowCenter) {Node};
        \coordinate[left=of secondRowCenter] (secondRowLeft);
        \coordinate[right=of secondRowCenter] (secondRowRight);
        \node[legendNode, right=of secondRowRight] {Node using LLVM-IR};

        \coordinate[below=of secondRowCenter] (thirdRowCenter);
        \coordinate[left=of thirdRowCenter] (thirdRowLeft);
        \coordinate[right=of thirdRowCenter] (thirdRowRight);
        \path[nonLlvmIrPath] (thirdRowLeft) to (thirdRowRight);
        \node[legendNode, right=of thirdRowRight] {Passing non-LLVM-IR};

        \node[nonLlvmIrNode, below=of thirdRowCenter] (fourthRowCenter) {Node};
        \coordinate[left=of fourthRowCenter] (fourthRowLeft);
        \coordinate[right=of fourthRowCenter] (fourthRowRight);
        \node[legendNode, right=of fourthRowRight] {Node not using LLVM-IR};
    \end{tikzpicture}
}

%Shortcuts
\newcommand{\cfg}{\ac{CFG} }
\newcommand{\scop}{\ac{SCoP} }
\newcommand{\scops}{\ac{SCoP}s }
\newcommand{\llvmir}{\ac{LLVM-IR} }
\newcommand{\llvm}{\ac{LLVM} }
\newcommand{\opt}{LLVM Optimizer }
\newcommand{\generator}{LLVM Code Generator }
\newcommand{\linker}{LLVM Linker }

%temporary commands while drafting
\newcommand{\draftnote}[1]{\textit{\textbf{\textcolor{red}{#1}}}}

% ****************************************************************************************************
% If you like the classicthesis, then I would appreciate a postcard.
% My address can be found in the file ClassicThesis.pdf. A collection
% of the postcards I received so far is available online at
% http://postcards.miede.de
% ****************************************************************************************************


% ****************************************************************************************************
% 0. Set the encoding of your files. UTF-8 is the only sensible encoding nowadays. If you can't read
% äöüßáéçèê∂åëæƒÏ€ then change the encoding setting in your editor, not the line below. If your editor
% does not support utf8 use another editor!
% ****************************************************************************************************
\PassOptionsToPackage{utf8}{inputenc}
\usepackage{inputenc}

% ****************************************************************************************************
% 1. Configure classicthesis for your needs here, e.g., remove "drafting" below
% in order to deactivate the time-stamp on the pages
% ****************************************************************************************************
\PassOptionsToPackage{eulerchapternumbers,listings,%drafting,%
        pdfspacing,%floatperchapter,%linedheaders,%
        dottedtoc,
        subfig,beramono,eulermath,parts}{classicthesis}
% ********************************************************************
% Available options for classicthesis.sty
% (see ClassicThesis.pdf for more information):
% drafting
% parts nochapters linedheaders
% eulerchapternumbers beramono eulermath pdfspacing minionprospacing
% tocaligned dottedtoc manychapters
% listings floatperchapter subfig
% ********************************************************************


% ****************************************************************************************************
% 2. Personal data and user ad-hoc commands
% ****************************************************************************************************
\newcommand{\myTitle}{A Classic Thesis Style\xspace}
\newcommand{\mySubtitle}{An Homage to The Elements of Typographic Style\xspace}
\newcommand{\myDegree}{Nahe am Bachelor\xspace}
\newcommand{\myName}{Stefan Huber\xspace}
\newcommand{\myProf}{Put name here\xspace}
\newcommand{\myOtherProf}{Put name here\xspace}
\newcommand{\mySupervisor}{Andreas Simbürger\xspace}
\newcommand{\myFaculty}{Fakultät für Informatik und Mathematik\xspace}
\newcommand{\myDepartment}{Department\xspace}
\newcommand{\myUni}{Universität Passau\xspace}
\newcommand{\myLocation}{Passau\xspace}
\newcommand{\myTime}{September 2017\xspace}
\newcommand{\myVersion}{version 1\xspace}

% ********************************************************************
% Setup, finetuning, and useful commands
% ********************************************************************
\newcounter{dummy} % necessary for correct hyperlinks (to index, bib, etc.)
\newlength{\abcd} % for ab..z string length calculation
\providecommand{\mLyX}{L\kern-.1667em\lower.25em\hbox{Y}\kern-.125emX\@}
\newcommand{\eg}{e.\,g.}
\newcommand{\Eg}{E.\,g.}
% ****************************************************************************************************


% ****************************************************************************************************
% 3. Loading some handy packages
% ****************************************************************************************************
% ********************************************************************
% Packages with options that might require adjustments
% ********************************************************************
\PassOptionsToPackage{british}{babel}   % change this to your language(s)
% Spanish languages need extra options in order to work with this template
%\PassOptionsToPackage{spanish,es-lcroman}{babel}
\usepackage{babel}

\usepackage{csquotes}
\PassOptionsToPackage{%
    %backend=biber, %instead of bibtex
    backend=bibtex8,bibencoding=ascii,%
    language=auto,%
    style=numeric-comp,%
    %style=authoryear-comp, % Author 1999, 2010
    %bibstyle=authoryear,dashed=false, % dashed: substitute rep. author with ---
    sorting=nyt, % name, year, title
    maxbibnames=10, % default: 3, et al.
    %backref=true,%
    natbib=true % natbib compatibility mode (\citep and \citet still work)
}{biblatex}
    \usepackage{biblatex}

\PassOptionsToPackage{fleqn}{amsmath}       % math environments and more by the AMS
    \usepackage{amsmath}

% ********************************************************************
% General useful packages
% ********************************************************************
\PassOptionsToPackage{T1}{fontenc} % T2A for cyrillics
    \usepackage{fontenc}
\usepackage{textcomp} % fix warning with missing font shapes
\usepackage{scrhack} % fix warnings when using KOMA with listings package
\usepackage{xspace} % to get the spacing after macros right
\usepackage{mparhack} % get marginpar right
\usepackage{fixltx2e} % fixes some LaTeX stuff --> since 2015 in the LaTeX kernel (see below)
%\usepackage[latest]{latexrelease} % will be used once available in more distributions (ISSUE #107)
\PassOptionsToPackage{printonlyused,smaller}{acronym}
    \usepackage{acronym} % nice macros for handling all acronyms in the thesis
    %\renewcommand{\bflabel}[1]{{#1}\hfill} % fix the list of acronyms --> no longer working
    %\renewcommand*{\acsfont}[1]{\textsc{#1}}
    \renewcommand*{\aclabelfont}[1]{\acsfont{#1}}
% ****************************************************************************************************


% ****************************************************************************************************
% 4. Setup floats: tables, (sub)figures, and captions
% ****************************************************************************************************
\usepackage{tabularx} % better tables
    \setlength{\extrarowheight}{3pt} % increase table row height
\newcommand{\tableheadline}[1]{\multicolumn{1}{c}{\spacedlowsmallcaps{#1}}}
\newcommand{\myfloatalign}{\centering} % to be used with each float for alignment
\usepackage{caption}
% Thanks to cgnieder and Claus Lahiri
% http://tex.stackexchange.com/questions/69349/spacedlowsmallcaps-in-caption-label
% [REMOVED DUE TO OTHER PROBLEMS, SEE ISSUE #82]
%\DeclareCaptionLabelFormat{smallcaps}{\bothIfFirst{#1}{~}\MakeTextLowercase{\textsc{#2}}}
%\captionsetup{font=small,labelformat=smallcaps} % format=hang,
\captionsetup{font=small} % format=hang,
\usepackage{subfig}
% ****************************************************************************************************


% ****************************************************************************************************
% 5. Setup code listings
% ****************************************************************************************************
\begin{comment}
    \usepackage{listings}
    %\lstset{emph={trueIndex,root},emphstyle=\color{BlueViolet}}%\underbar} % for special keywords
    \lstset{language=C++,
        morekeywords={PassOptionsToPackage,selectlanguage},
        keywordstyle=\color{NavyBlue},%\bfseries,
        basicstyle=\small\ttfamily,
        %identifierstyle=\color{NavyBlue},
        commentstyle=\color{Green}\ttfamily,
        stringstyle=\rmfamily,
        numbers=left,
        numberstyle=\scriptsize,%\tiny
        %stepnumber=5,
        numbersep=8pt,
        showstringspaces=false,
        breaklines=true,
        %frameround=ftff,
        %frame=single,
        belowcaptionskip=.75\baselineskip,
        frame=L
    }
\end{comment}
% ****************************************************************************************************


% ****************************************************************************************************
% 6. PDFLaTeX, hyperreferences and citation backreferences
% ****************************************************************************************************
% ********************************************************************
% Using PDFLaTeX
% ********************************************************************
\PassOptionsToPackage{pdftex,hyperfootnotes=false,pdfpagelabels}{hyperref}
    \usepackage{hyperref}  % backref linktocpage pagebackref
\pdfcompresslevel=9
\pdfadjustspacing=1
\PassOptionsToPackage{pdftex}{graphicx}
    \usepackage{graphicx}


% ********************************************************************
% Hyperreferences
% ********************************************************************
\hypersetup{%
    %draft, % = no hyperlinking at all (useful in b/w printouts)
    colorlinks=true, linktocpage=true, pdfstartpage=3, pdfstartview=FitV,%
    % uncomment the following line if you want to have black links (e.g., for printing)
    %colorlinks=false, linktocpage=false, pdfstartpage=3, pdfstartview=FitV, pdfborder={0 0 0},%
    breaklinks=true, pdfpagemode=UseNone, pageanchor=true, pdfpagemode=UseOutlines,%
    plainpages=false, bookmarksnumbered, bookmarksopen=true, bookmarksopenlevel=1,%
    hypertexnames=true, pdfhighlight=/O,%nesting=true,%frenchlinks,%
    urlcolor=webbrown, linkcolor=RoyalBlue, citecolor=webgreen, %pagecolor=RoyalBlue,%
    %urlcolor=Black, linkcolor=Black, citecolor=Black, %pagecolor=Black,%
    pdftitle={\myTitle},%
    pdfauthor={\textcopyright\ \myName, \myUni, \myFaculty},%
    pdfsubject={},%
    pdfkeywords={},%
    pdfcreator={pdfLaTeX},%
    pdfproducer={LaTeX with hyperref and classicthesis}%
}

% ********************************************************************
% Setup autoreferences
% ********************************************************************
% There are some issues regarding autorefnames
% http://www.ureader.de/msg/136221647.aspx
% http://www.tex.ac.uk/cgi-bin/texfaq2html?label=latexwords
% you have to redefine the makros for the
% language you use, e.g., american, ngerman
% (as chosen when loading babel/AtBeginDocument)
% ********************************************************************
\makeatletter
\@ifpackageloaded{babel}%
    {%
        \addto\extrasamerican{%
            \renewcommand*{\figureautorefname}{Figure}%
            \renewcommand*{\tableautorefname}{Table}%
            \renewcommand*{\partautorefname}{Part}%
            \renewcommand*{\chapterautorefname}{Chapter}%
            \renewcommand*{\sectionautorefname}{Section}%
            \renewcommand*{\subsectionautorefname}{Section}%
            \renewcommand*{\subsubsectionautorefname}{Section}%
        }%
        \addto\extrasbritish{%
            \renewcommand*{\figureautorefname}{Figure}%
            \renewcommand*{\tableautorefname}{Table}%
            \renewcommand*{\partautorefname}{Part}%
            \renewcommand*{\chapterautorefname}{Chapter}%
            \renewcommand*{\sectionautorefname}{Section}%
            \renewcommand*{\subsectionautorefname}{Section}%
            \renewcommand*{\subsubsectionautorefname}{Section}%
        }%
        \addto\extrasngerman{%
            \renewcommand*{\paragraphautorefname}{Absatz}%
            \renewcommand*{\subparagraphautorefname}{Unterabsatz}%
            \renewcommand*{\footnoteautorefname}{Fu\"snote}%
            \renewcommand*{\FancyVerbLineautorefname}{Zeile}%
            \renewcommand*{\theoremautorefname}{Theorem}%
            \renewcommand*{\appendixautorefname}{Anhang}%
            \renewcommand*{\equationautorefname}{Gleichung}%
            \renewcommand*{\itemautorefname}{Punkt}%
        }%
        % Fix to getting autorefs for subfigures right (thanks to Belinda Vogt for changing the definition)
        \providecommand{\subfigureautorefname}{\figureautorefname}%
    }{\relax}
\makeatother


% ****************************************************************************************************
% 7. Last calls before the bar closes
% ****************************************************************************************************
% ********************************************************************
% Development Stuff
% ********************************************************************
\listfiles
%\PassOptionsToPackage{l2tabu,orthodox,abort}{nag}
%   \usepackage{nag}
%\PassOptionsToPackage{warning, all}{onlyamsmath}
%   \usepackage{onlyamsmath}

% ********************************************************************
% Last, but not least...
% ********************************************************************
\usepackage{classicthesis}
% ****************************************************************************************************


% ****************************************************************************************************
% 8. Further adjustments (experimental)
% ****************************************************************************************************
% ********************************************************************
% Changing the text area
% ********************************************************************
%\linespread{1.05} % a bit more for Palatino
%\areaset[current]{312pt}{761pt} % 686 (factor 2.2) + 33 head + 42 head \the\footskip
%\setlength{\marginparwidth}{7em}%
%\setlength{\marginparsep}{2em}%

% ********************************************************************
% Using different fonts
% ********************************************************************
%\usepackage[oldstylenums]{kpfonts} % oldstyle notextcomp
%\usepackage[osf]{libertine}
%\usepackage[light,condensed,math]{iwona}
%\renewcommand{\sfdefault}{iwona}
%\usepackage{lmodern} % <-- no osf support :-(
%\usepackage{cfr-lm} %
%\usepackage[urw-garamond]{mathdesign} <-- no osf support :-(
%\usepackage[default,osfigures]{opensans} % scale=0.95
%\usepackage[sfdefault]{FiraSans}
% ****************************************************************************************************


%Customization Stefan Huber
%custom colors
\definecolor{defaultColor}{HTML}{1E90FF}
\definecolor{llvmIrColor}{HTML}{228B22}
\definecolor{nonLlvmIrColor}{HTML}{B22222}

%minted configure
\setminted{
    linenos,
    breaklines
}
\providecommand*{\listingautorefname}{Listing}
\newenvironment{code}{\captionsetup{type=listing}}{}
%\SetupFloatingEnvironment{listing}{name=Program code} %newfloat package
%\renewcommand{\listoflistingscaption}{List of Program Code} %float package

%Tikz
\usepackage{tikz}
\usetikzlibrary{arrows.meta, positioning, shapes, snakes, fit}
\tikzset{
    node distance = 5mm and 15mm,
    defaultNode/.style = {rectangle, rounded corners, draw=gray, drop shadow, very thick, bottom color=defaultColor!50!white, text centered, top color=white, minimum height=1.2em, minimum width=3em},
    llvmIrNode/.style = {defaultNode, bottom color=llvmIrColor!50!white},
    nonLlvmIrNode/.style = {defaultNode, bottom color=nonLlvmIrColor!50!white},
    legendNode/.style = {rectangle},
    defaultPath/.style = {-{Stealth[length=4mm]}, color=defaultColor, line width=0.2em, draw, rounded corners},
    llvmIrPath/.style = {defaultPath, color=llvmIrColor},
    nonLlvmIrPath/.style = {defaultPath, color=nonLlvmIrColor}
}
\newcommand{\legend}[0]{
    \begin{tikzpicture}
        \coordinate (firstRowCenter);
        \coordinate[left=of firstRowCenter] (firstRowLeft);
        \coordinate[right=of firstRowCenter] (firstRowRight);
        \path[llvmIrPath] (firstRowLeft) to (firstRowRight);
        \node[legendNode, right=of firstRowRight] {Passing LLVM-IR};

        \node[llvmIrNode, below=of firstRowCenter] (secondRowCenter) {Node};
        \coordinate[left=of secondRowCenter] (secondRowLeft);
        \coordinate[right=of secondRowCenter] (secondRowRight);
        \node[legendNode, right=of secondRowRight] {Node using LLVM-IR};

        \coordinate[below=of secondRowCenter] (thirdRowCenter);
        \coordinate[left=of thirdRowCenter] (thirdRowLeft);
        \coordinate[right=of thirdRowCenter] (thirdRowRight);
        \path[nonLlvmIrPath] (thirdRowLeft) to (thirdRowRight);
        \node[legendNode, right=of thirdRowRight] {Passing non-LLVM-IR};

        \node[nonLlvmIrNode, below=of thirdRowCenter] (fourthRowCenter) {Node};
        \coordinate[left=of fourthRowCenter] (fourthRowLeft);
        \coordinate[right=of fourthRowCenter] (fourthRowRight);
        \node[legendNode, right=of fourthRowRight] {Node not using LLVM-IR};
    \end{tikzpicture}
}

%Shortcuts
\newcommand{\cfg}{\ac{CFG} }
\newcommand{\scop}{\ac{SCoP} }
\newcommand{\scops}{\ac{SCoP}s }
\newcommand{\llvmir}{\ac{LLVM-IR} }
\newcommand{\llvm}{\ac{LLVM} }
\newcommand{\opt}{LLVM Optimizer }
\newcommand{\generator}{LLVM Code Generator }
\newcommand{\linker}{LLVM Linker }

%temporary commands while drafting
\newcommand{\draftnote}[1]{\textit{\textbf{\textcolor{red}{#1}}}}

% ****************************************************************************************************
% If you like the classicthesis, then I would appreciate a postcard.
% My address can be found in the file ClassicThesis.pdf. A collection
% of the postcards I received so far is available online at
% http://postcards.miede.de
% ****************************************************************************************************


% ****************************************************************************************************
% 0. Set the encoding of your files. UTF-8 is the only sensible encoding nowadays. If you can't read
% äöüßáéçèê∂åëæƒÏ€ then change the encoding setting in your editor, not the line below. If your editor
% does not support utf8 use another editor!
% ****************************************************************************************************
\PassOptionsToPackage{utf8}{inputenc}
\usepackage{inputenc}

% ****************************************************************************************************
% 1. Configure classicthesis for your needs here, e.g., remove "drafting" below
% in order to deactivate the time-stamp on the pages
% ****************************************************************************************************
\PassOptionsToPackage{eulerchapternumbers,listings,%drafting,%
        pdfspacing,%floatperchapter,%linedheaders,%
        dottedtoc,
        subfig,beramono,eulermath,parts}{classicthesis}
% ********************************************************************
% Available options for classicthesis.sty
% (see ClassicThesis.pdf for more information):
% drafting
% parts nochapters linedheaders
% eulerchapternumbers beramono eulermath pdfspacing minionprospacing
% tocaligned dottedtoc manychapters
% listings floatperchapter subfig
% ********************************************************************


% ****************************************************************************************************
% 2. Personal data and user ad-hoc commands
% ****************************************************************************************************
\newcommand{\myTitle}{A Classic Thesis Style\xspace}
\newcommand{\mySubtitle}{An Homage to The Elements of Typographic Style\xspace}
\newcommand{\myDegree}{Nahe am Bachelor\xspace}
\newcommand{\myName}{Stefan Huber\xspace}
\newcommand{\myEmail}{stefan.huber.niedling@outlook.com\xspace}
\newcommand{\myProf}{Christian Lengauer\xspace}
\newcommand{\myOtherProf}{Put name here\xspace}
\newcommand{\mySupervisor}{Andreas Simbürger\xspace}
\newcommand{\myFaculty}{Fakultät für Informatik und Mathematik\xspace}
\newcommand{\myDepartment}{Department\xspace}
\newcommand{\myUni}{Universität Passau\xspace}
\newcommand{\myLocation}{Passau\xspace}
\newcommand{\myTime}{September 2017\xspace}
\newcommand{\myVersion}{version 1\xspace}

% ********************************************************************
% Setup, finetuning, and useful commands
% ********************************************************************
\newcounter{dummy} % necessary for correct hyperlinks (to index, bib, etc.)
\newlength{\abcd} % for ab..z string length calculation
\providecommand{\mLyX}{L\kern-.1667em\lower.25em\hbox{Y}\kern-.125emX\@}
\newcommand{\eg}{e.\,g. }
\newcommand{\Eg}{E.\,g. }
% ****************************************************************************************************


% ****************************************************************************************************
% 3. Loading some handy packages
% ****************************************************************************************************
% ********************************************************************
% Packages with options that might require adjustments
% ********************************************************************
\PassOptionsToPackage{british}{babel}   % change this to your language(s)
% Spanish languages need extra options in order to work with this template
%\PassOptionsToPackage{spanish,es-lcroman}{babel}
\usepackage{babel}

\usepackage{csquotes}
\PassOptionsToPackage{%
    %backend=biber, %instead of bibtex
    backend=bibtex8,bibencoding=ascii,%
    language=auto,%
    style=numeric-comp,%
    %style=authoryear-comp, % Author 1999, 2010
    %bibstyle=authoryear,dashed=false, % dashed: substitute rep. author with ---
    sorting=nyt, % name, year, title
    maxbibnames=10, % default: 3, et al.
    %backref=true,%
    natbib=true % natbib compatibility mode (\citep and \citet still work)
}{biblatex}
    \usepackage{biblatex}

\PassOptionsToPackage{fleqn}{amsmath}       % math environments and more by the AMS
    \usepackage{amsmath}

% ********************************************************************
% General useful packages
% ********************************************************************
\PassOptionsToPackage{T1}{fontenc} % T2A for cyrillics
    \usepackage{fontenc}
\usepackage{textcomp} % fix warning with missing font shapes
\usepackage{scrhack} % fix warnings when using KOMA with listings package
\usepackage{xspace} % to get the spacing after macros right
\usepackage{mparhack} % get marginpar right
\usepackage{fixltx2e} % fixes some LaTeX stuff --> since 2015 in the LaTeX kernel (see below)
%\usepackage[latest]{latexrelease} % will be used once available in more distributions (ISSUE #107)
\PassOptionsToPackage{printonlyused,smaller}{acronym}
    \usepackage{acronym} % nice macros for handling all acronyms in the thesis
    %\renewcommand{\bflabel}[1]{{#1}\hfill} % fix the list of acronyms --> no longer working
    %\renewcommand*{\acsfont}[1]{\textsc{#1}}
    \renewcommand*{\aclabelfont}[1]{\acsfont{#1}}
% ****************************************************************************************************


% ****************************************************************************************************
% 4. Setup floats: tables, (sub)figures, and captions
% ****************************************************************************************************
\usepackage{tabularx} % better tables
    \setlength{\extrarowheight}{3pt} % increase table row height
\newcommand{\tableheadline}[1]{\multicolumn{1}{c}{\spacedlowsmallcaps{#1}}}
\newcommand{\myfloatalign}{\centering} % to be used with each float for alignment
\usepackage{caption}
% Thanks to cgnieder and Claus Lahiri
% http://tex.stackexchange.com/questions/69349/spacedlowsmallcaps-in-caption-label
% [REMOVED DUE TO OTHER PROBLEMS, SEE ISSUE #82]
%\DeclareCaptionLabelFormat{smallcaps}{\bothIfFirst{#1}{~}\MakeTextLowercase{\textsc{#2}}}
%\captionsetup{font=small,labelformat=smallcaps} % format=hang,
\captionsetup{font=small} % format=hang,
\usepackage{subfig}
% ****************************************************************************************************


% ****************************************************************************************************
% 5. Setup code listings
% ****************************************************************************************************
\begin{comment}
    \usepackage{listings}
    %\lstset{emph={trueIndex,root},emphstyle=\color{BlueViolet}}%\underbar} % for special keywords
    \lstset{language=C++,
        morekeywords={PassOptionsToPackage,selectlanguage},
        keywordstyle=\color{NavyBlue},%\bfseries,
        basicstyle=\small\ttfamily,
        %identifierstyle=\color{NavyBlue},
        commentstyle=\color{Green}\ttfamily,
        stringstyle=\rmfamily,
        numbers=left,
        numberstyle=\scriptsize,%\tiny
        %stepnumber=5,
        numbersep=8pt,
        showstringspaces=false,
        breaklines=true,
        %frameround=ftff,
        %frame=single,
        belowcaptionskip=.75\baselineskip,
        frame=L
    }
\end{comment}
% ****************************************************************************************************


% ****************************************************************************************************
% 6. PDFLaTeX, hyperreferences and citation backreferences
% ****************************************************************************************************
% ********************************************************************
% Using PDFLaTeX
% ********************************************************************
\PassOptionsToPackage{pdftex,hyperfootnotes=false,pdfpagelabels}{hyperref}
    \usepackage{hyperref}  % backref linktocpage pagebackref
\pdfcompresslevel=9
\pdfadjustspacing=1
\PassOptionsToPackage{pdftex}{graphicx}
    \usepackage{graphicx}


% ********************************************************************
% Hyperreferences
% ********************************************************************
\hypersetup{%
    %draft, % = no hyperlinking at all (useful in b/w printouts)
    colorlinks=true, linktocpage=true, pdfstartpage=3, pdfstartview=FitV,%
    % uncomment the following line if you want to have black links (e.g., for printing)
    %colorlinks=false, linktocpage=false, pdfstartpage=3, pdfstartview=FitV, pdfborder={0 0 0},%
    breaklinks=true, pdfpagemode=UseNone, pageanchor=true, pdfpagemode=UseOutlines,%
    plainpages=false, bookmarksnumbered, bookmarksopen=true, bookmarksopenlevel=1,%
    hypertexnames=true, pdfhighlight=/O,%nesting=true,%frenchlinks,%
    urlcolor=webbrown, linkcolor=RoyalBlue, citecolor=webgreen, %pagecolor=RoyalBlue,%
    %urlcolor=Black, linkcolor=Black, citecolor=Black, %pagecolor=Black,%
    pdftitle={\myTitle},%
    pdfauthor={\textcopyright\ \myName, \myUni, \myFaculty},%
    pdfsubject={},%
    pdfkeywords={},%
    pdfcreator={pdfLaTeX},%
    pdfproducer={LaTeX with hyperref and classicthesis}%
}

% ********************************************************************
% Setup autoreferences
% ********************************************************************
% There are some issues regarding autorefnames
% http://www.ureader.de/msg/136221647.aspx
% http://www.tex.ac.uk/cgi-bin/texfaq2html?label=latexwords
% you have to redefine the makros for the
% language you use, e.g., american, ngerman
% (as chosen when loading babel/AtBeginDocument)
% ********************************************************************
\makeatletter
\@ifpackageloaded{babel}%
    {%
        \addto\extrasamerican{%
            \renewcommand*{\figureautorefname}{Figure}%
            \renewcommand*{\tableautorefname}{Table}%
            \renewcommand*{\partautorefname}{Part}%
            \renewcommand*{\chapterautorefname}{Chapter}%
            \renewcommand*{\sectionautorefname}{Section}%
            \renewcommand*{\subsectionautorefname}{Section}%
            \renewcommand*{\subsubsectionautorefname}{Section}%
        }%
        \addto\extrasbritish{%
            \renewcommand*{\figureautorefname}{Figure}%
            \renewcommand*{\tableautorefname}{Table}%
            \renewcommand*{\partautorefname}{Part}%
            \renewcommand*{\chapterautorefname}{Chapter}%
            \renewcommand*{\sectionautorefname}{Section}%
            \renewcommand*{\subsectionautorefname}{Section}%
            \renewcommand*{\subsubsectionautorefname}{Section}%
            \providecommand*{\listingsautorefname}{Listing}%
            \providecommand*{\listingsname}{Listing}%
        }%
        \addto\extrasngerman{%
            \renewcommand*{\paragraphautorefname}{Absatz}%
            \renewcommand*{\subparagraphautorefname}{Unterabsatz}%
            \renewcommand*{\footnoteautorefname}{Fu\"snote}%
            \renewcommand*{\FancyVerbLineautorefname}{Zeile}%
            \renewcommand*{\theoremautorefname}{Theorem}%
            \renewcommand*{\appendixautorefname}{Anhang}%
            \renewcommand*{\equationautorefname}{Gleichung}%
            \renewcommand*{\itemautorefname}{Punkt}%
        }%
        % Fix to getting autorefs for subfigures right (thanks to Belinda Vogt for changing the definition)
        \providecommand{\subfigureautorefname}{\figureautorefname}%
    }{\relax}
\makeatother


% ****************************************************************************************************
% 7. Last calls before the bar closes
% ****************************************************************************************************
% ********************************************************************
% Development Stuff
% ********************************************************************
\listfiles
%\PassOptionsToPackage{l2tabu,orthodox,abort}{nag}
%   \usepackage{nag}
%\PassOptionsToPackage{warning, all}{onlyamsmath}
%   \usepackage{onlyamsmath}

% ********************************************************************
% Last, but not least...
% ********************************************************************
\usepackage{classicthesis}
% ****************************************************************************************************


% ****************************************************************************************************
% 8. Further adjustments (experimental)
% ****************************************************************************************************
% ********************************************************************
% Changing the text area
% ********************************************************************
%\linespread{1.05} % a bit more for Palatino
%\areaset[current]{312pt}{761pt} % 686 (factor 2.2) + 33 head + 42 head \the\footskip
%\setlength{\marginparwidth}{7em}%
%\setlength{\marginparsep}{2em}%

% ********************************************************************
% Using different fonts
% ********************************************************************
%\usepackage[oldstylenums]{kpfonts} % oldstyle notextcomp
%\usepackage[osf]{libertine}
%\usepackage[light,condensed,math]{iwona}
%\renewcommand{\sfdefault}{iwona}
%\usepackage{lmodern} % <-- no osf support :-(
%\usepackage{cfr-lm} %
%\usepackage[urw-garamond]{mathdesign} <-- no osf support :-(
%\usepackage[default,osfigures]{opensans} % scale=0.95
%\usepackage[sfdefault]{FiraSans}
% ****************************************************************************************************


%Customization Stefan Huber
%custom colors
\definecolor{defaultColor}{HTML}{1E90FF}
\definecolor{llvmIrColor}{HTML}{228B22}
\definecolor{nonLlvmIrColor}{HTML}{B22222}

%minted configure
\setminted{
    linenos,
    breaklines
}
\newenvironment{code}{\captionsetup{type=listings}}{}

%Tikz
\usepackage{tikz}
\usetikzlibrary{arrows.meta, positioning, shapes, snakes, fit}
\tikzset{
    node distance = 5mm and 15mm,
    defaultNode/.style = {rectangle, rounded corners, draw=gray, drop shadow, very thick, bottom color=defaultColor!50!white, text centered, top color=white, minimum height=1.2em, minimum width=3em},
    llvmIrNode/.style = {defaultNode, bottom color=llvmIrColor!50!white},
    nonLlvmIrNode/.style = {defaultNode, bottom color=nonLlvmIrColor!50!white},
    legendNode/.style = {rectangle},
    defaultPath/.style = {-{Stealth[length=4mm]}, color=defaultColor, line width=0.2em, draw, rounded corners},
    llvmIrPath/.style = {defaultPath, color=llvmIrColor},
    nonLlvmIrPath/.style = {defaultPath, color=nonLlvmIrColor}
}
\newcommand{\legend}[0]{
    \begin{tikzpicture}
        \coordinate (firstRowCenter);
        \coordinate[left=of firstRowCenter] (firstRowLeft);
        \coordinate[right=of firstRowCenter] (firstRowRight);
        \path[llvmIrPath] (firstRowLeft) to (firstRowRight);
        \node[legendNode, right=of firstRowRight] {Passing LLVM-IR};

        \node[llvmIrNode, below=of firstRowCenter] (secondRowCenter) {Node};
        \coordinate[left=of secondRowCenter] (secondRowLeft);
        \coordinate[right=of secondRowCenter] (secondRowRight);
        \node[legendNode, right=of secondRowRight] {Node using LLVM-IR};

        \coordinate[below=of secondRowCenter] (thirdRowCenter);
        \coordinate[left=of thirdRowCenter] (thirdRowLeft);
        \coordinate[right=of thirdRowCenter] (thirdRowRight);
        \path[nonLlvmIrPath] (thirdRowLeft) to (thirdRowRight);
        \node[legendNode, right=of thirdRowRight] {Passing non-LLVM-IR};

        \node[nonLlvmIrNode, below=of thirdRowCenter] (fourthRowCenter) {Node};
        \coordinate[left=of fourthRowCenter] (fourthRowLeft);
        \coordinate[right=of fourthRowCenter] (fourthRowRight);
        \node[legendNode, right=of fourthRowRight] {Node not using LLVM-IR};
    \end{tikzpicture}
}

%tabularx
\newcolumntype{C}{>{\centering\arraybackslash}X}

%Shortcuts
\newcommand{\cfg}{\ac{CFG} }
\newcommand{\ir}{\ac{IR} }
\newcommand{\scop}{\ac{SCoP} }
\newcommand{\scops}{\ac{SCoP}s }
\newcommand{\llvmir}{\ac{LLVM-IR} }
\newcommand{\llvm}{\ac{LLVM} }
\newcommand{\opt}{LLVM Optimizer }
\newcommand{\generator}{LLVM Code Generator }
\newcommand{\linker}{LLVM Linker }

%temporary commands while drafting
\newcommand{\draftnote}[1]{\textit{\textbf{\textcolor{red}{#1}}}}
