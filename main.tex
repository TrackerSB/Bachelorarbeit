\documentclass[a4paper, ngerman]{scrreprt}
\usepackage[ngerman]{babel}
\usepackage[utf8]{inputenc}
\usepackage{hyperref}
\usepackage[backend=biber, style=alphabetic, citestyle=alphabetic-verb]{biblatex}
\usepackage[babel,german=guillemets]{csquotes}
\usepackage{comment}
\usepackage{listings}
\usepackage{color}

\lstset{basicstyle=\footnotesize, numbers=left, numberstyle=\tiny, keywordstyle=\color{blue}, numbersep=5pt, language=c++, breaklines=true, tabsize=4, keepspaces=true}

\addbibresource{references.bib}

\begin{document}

%%%%%%%%%%%%%%%%%%%%%%%%%%%%%%%%%%%%%%%%%
% University Assignment Title Page 
% LaTeX Template
% Version 1.0 (27/12/12)
%
% This template has been downloaded from:
% http://www.LaTeXTemplates.com
%
% Original author:
% WikiBooks (http://en.wikibooks.org/wiki/LaTeX/Title_Creation)
%
% License:
% CC BY-NC-SA 3.0 (http://creativecommons.org/licenses/by-nc-sa/3.0/)
% 
% Instructions for using this template:
% This title page is capable of being compiled as is. This is not useful for 
% including it in another document. To do this, you have two options: 
%
% 1) Copy/paste everything between \begin{document} and \end{document} 
% starting at \begin{titlepage} and paste this into another LaTeX file where you 
% want your title page.
% OR
% 2) Remove everything outside the \begin{titlepage} and \end{titlepage} and 
% move this file to the same directory as the LaTeX file you wish to add it to. 
% Then add %%%%%%%%%%%%%%%%%%%%%%%%%%%%%%%%%%%%%%%%%
% University Assignment Title Page 
% LaTeX Template
% Version 1.0 (27/12/12)
%
% This template has been downloaded from:
% http://www.LaTeXTemplates.com
%
% Original author:
% WikiBooks (http://en.wikibooks.org/wiki/LaTeX/Title_Creation)
%
% License:
% CC BY-NC-SA 3.0 (http://creativecommons.org/licenses/by-nc-sa/3.0/)
% 
% Instructions for using this template:
% This title page is capable of being compiled as is. This is not useful for 
% including it in another document. To do this, you have two options: 
%
% 1) Copy/paste everything between \begin{document} and \end{document} 
% starting at \begin{titlepage} and paste this into another LaTeX file where you 
% want your title page.
% OR
% 2) Remove everything outside the \begin{titlepage} and \end{titlepage} and 
% move this file to the same directory as the LaTeX file you wish to add it to. 
% Then add %%%%%%%%%%%%%%%%%%%%%%%%%%%%%%%%%%%%%%%%%
% University Assignment Title Page 
% LaTeX Template
% Version 1.0 (27/12/12)
%
% This template has been downloaded from:
% http://www.LaTeXTemplates.com
%
% Original author:
% WikiBooks (http://en.wikibooks.org/wiki/LaTeX/Title_Creation)
%
% License:
% CC BY-NC-SA 3.0 (http://creativecommons.org/licenses/by-nc-sa/3.0/)
% 
% Instructions for using this template:
% This title page is capable of being compiled as is. This is not useful for 
% including it in another document. To do this, you have two options: 
%
% 1) Copy/paste everything between \begin{document} and \end{document} 
% starting at \begin{titlepage} and paste this into another LaTeX file where you 
% want your title page.
% OR
% 2) Remove everything outside the \begin{titlepage} and \end{titlepage} and 
% move this file to the same directory as the LaTeX file you wish to add it to. 
% Then add \input{./title_page_1.tex} to your LaTeX file where you want your
% title page.
%
%%%%%%%%%%%%%%%%%%%%%%%%%%%%%%%%%%%%%%%%%

%----------------------------------------------------------------------------------------
%	PACKAGES AND OTHER DOCUMENT CONFIGURATIONS
%----------------------------------------------------------------------------------------

%\documentclass[12pt]{article}

%\begin{document}

\begin{titlepage}

\newcommand{\HRule}{\rule{\linewidth}{0.5mm}} % Defines a new command for the horizontal lines, change thickness here

\center % Center everything on the page
 
%----------------------------------------------------------------------------------------
%	HEADING SECTIONS
%----------------------------------------------------------------------------------------

\textsc{\LARGE Universität Passau}\\[1.5cm] % Name of your university/college
\textsc{\Large Major Heading}\\[0.5cm] % Major heading such as course name
\textsc{\large Minor Heading}\\[0.5cm] % Minor heading such as course title

%----------------------------------------------------------------------------------------
%	TITLE SECTION
%----------------------------------------------------------------------------------------

\HRule \\[0.4cm]
{ \huge \bfseries Title}\\[0.4cm] % Title of your document
\HRule \\[1.5cm]
 
%----------------------------------------------------------------------------------------
%	AUTHOR SECTION
%----------------------------------------------------------------------------------------

\begin{minipage}{0.4\textwidth}
\begin{flushleft} \large
\emph{Author:}\\
Stefan \textsc{Huber} % Your name
\end{flushleft}
\end{minipage}
~
\begin{minipage}{0.4\textwidth}
\begin{flushright} \large
\emph{Betreuer:} \\
Andreas \textsc{Simbürger} % Supervisor's Name
\end{flushright}
\end{minipage}\\[4cm]

% If you don't want a supervisor, uncomment the two lines below and remove the section above
%\Large \emph{Author:}\\
%John \textsc{Smith}\\[3cm] % Your name

%----------------------------------------------------------------------------------------
%	DATE SECTION
%----------------------------------------------------------------------------------------

{\large \today}\\[3cm] % Date, change the \today to a set date if you want to be precise

%----------------------------------------------------------------------------------------
%	LOGO SECTION
%----------------------------------------------------------------------------------------

%\includegraphics{Logo}\\[1cm] % Include a department/university logo - this will require the graphicx package
 
%----------------------------------------------------------------------------------------

\vfill % Fill the rest of the page with whitespace

\end{titlepage}
%\end{document}\grid
 to your LaTeX file where you want your
% title page.
%
%%%%%%%%%%%%%%%%%%%%%%%%%%%%%%%%%%%%%%%%%

%----------------------------------------------------------------------------------------
%	PACKAGES AND OTHER DOCUMENT CONFIGURATIONS
%----------------------------------------------------------------------------------------

%\documentclass[12pt]{article}

%\begin{document}

\begin{titlepage}

\newcommand{\HRule}{\rule{\linewidth}{0.5mm}} % Defines a new command for the horizontal lines, change thickness here

\center % Center everything on the page
 
%----------------------------------------------------------------------------------------
%	HEADING SECTIONS
%----------------------------------------------------------------------------------------

\textsc{\LARGE Universität Passau}\\[1.5cm] % Name of your university/college
\textsc{\Large Major Heading}\\[0.5cm] % Major heading such as course name
\textsc{\large Minor Heading}\\[0.5cm] % Minor heading such as course title

%----------------------------------------------------------------------------------------
%	TITLE SECTION
%----------------------------------------------------------------------------------------

\HRule \\[0.4cm]
{ \huge \bfseries Title}\\[0.4cm] % Title of your document
\HRule \\[1.5cm]
 
%----------------------------------------------------------------------------------------
%	AUTHOR SECTION
%----------------------------------------------------------------------------------------

\begin{minipage}{0.4\textwidth}
\begin{flushleft} \large
\emph{Author:}\\
Stefan \textsc{Huber} % Your name
\end{flushleft}
\end{minipage}
~
\begin{minipage}{0.4\textwidth}
\begin{flushright} \large
\emph{Betreuer:} \\
Andreas \textsc{Simbürger} % Supervisor's Name
\end{flushright}
\end{minipage}\\[4cm]

% If you don't want a supervisor, uncomment the two lines below and remove the section above
%\Large \emph{Author:}\\
%John \textsc{Smith}\\[3cm] % Your name

%----------------------------------------------------------------------------------------
%	DATE SECTION
%----------------------------------------------------------------------------------------

{\large \today}\\[3cm] % Date, change the \today to a set date if you want to be precise

%----------------------------------------------------------------------------------------
%	LOGO SECTION
%----------------------------------------------------------------------------------------

%\includegraphics{Logo}\\[1cm] % Include a department/university logo - this will require the graphicx package
 
%----------------------------------------------------------------------------------------

\vfill % Fill the rest of the page with whitespace

\end{titlepage}
%\end{document}\grid
 to your LaTeX file where you want your
% title page.
%
%%%%%%%%%%%%%%%%%%%%%%%%%%%%%%%%%%%%%%%%%

%----------------------------------------------------------------------------------------
%	PACKAGES AND OTHER DOCUMENT CONFIGURATIONS
%----------------------------------------------------------------------------------------

%\documentclass[12pt]{article}

%\begin{document}

\begin{titlepage}

\newcommand{\HRule}{\rule{\linewidth}{0.5mm}} % Defines a new command for the horizontal lines, change thickness here

\center % Center everything on the page
 
%----------------------------------------------------------------------------------------
%	HEADING SECTIONS
%----------------------------------------------------------------------------------------

\textsc{\LARGE Universität Passau}\\[1.5cm] % Name of your university/college
\textsc{\Large Major Heading}\\[0.5cm] % Major heading such as course name
\textsc{\large Minor Heading}\\[0.5cm] % Minor heading such as course title

%----------------------------------------------------------------------------------------
%	TITLE SECTION
%----------------------------------------------------------------------------------------

\HRule \\[0.4cm]
{ \huge \bfseries Title}\\[0.4cm] % Title of your document
\HRule \\[1.5cm]
 
%----------------------------------------------------------------------------------------
%	AUTHOR SECTION
%----------------------------------------------------------------------------------------

\begin{minipage}{0.4\textwidth}
\begin{flushleft} \large
\emph{Author:}\\
Stefan \textsc{Huber} % Your name
\end{flushleft}
\end{minipage}
~
\begin{minipage}{0.4\textwidth}
\begin{flushright} \large
\emph{Betreuer:} \\
Andreas \textsc{Simbürger} % Supervisor's Name
\end{flushright}
\end{minipage}\\[4cm]

% If you don't want a supervisor, uncomment the two lines below and remove the section above
%\Large \emph{Author:}\\
%John \textsc{Smith}\\[3cm] % Your name

%----------------------------------------------------------------------------------------
%	DATE SECTION
%----------------------------------------------------------------------------------------

{\large \today}\\[3cm] % Date, change the \today to a set date if you want to be precise

%----------------------------------------------------------------------------------------
%	LOGO SECTION
%----------------------------------------------------------------------------------------

%\includegraphics{Logo}\\[1cm] % Include a department/university logo - this will require the graphicx package
 
%----------------------------------------------------------------------------------------

\vfill % Fill the rest of the page with whitespace

\end{titlepage}
%\end{document}\grid

\tableofcontents

\chapter{Motivation}
\chapter{Motivation}
As the number of cores of processors and the variety of devices and platforms is increasing there is a need for an efficient compiler infrastructure which is capable of performing language and platform independent optimizations allowing to utilize the capacity of the underlying hardware as much as possible.
A popular approach of such a compiler infrastructure is LLVM. \cite{LLVMUsers}
This infrastructure has options to extend it by adding plugins like Polly.
Polly is a plugin implementing loop optimizations on the basis of the polyhedral model.
When using Polly there arises the question about the coverage of the parts which can currently be automatically optimized, what are the common reasons of these parts for not being even bigger and the theoretical speedup.\\
Answering these questions gives on the one hand an assessment to whether Polly is already able to optimize a high percentage of the instructions of given programs -- commonly used and installed -- and on the other hand a potential to investigate the main reasons for the automatically optimizable parts not being as big as expected and the theoretical impact when being able eliminating specific reasons.


\chapter{Polyeder-Modell}
\section{Ansatz}
\section{Repräsentation}


\chapter{Begriffsklärung}
\section{Region}
\section{SCoP}
\section{Pass}


\chapter{Aufbau LLVM}
\section{Lifecycle}
\section{LLVM-IR}


\begin{comment}
Background:
Pipeline, Geschichte, Standardpaper auf LLVM Website

polly.llvm.org => Documentation Pipeline
Auch für Polly

Wie funktioniert polly ansich?
RegionInfo Definition Region
Journal-Artikel (Lengauer, Grosser)

Compilerbau (Drachenbuch) Grundlagen (Dominanz)
Compilers alfred v aho, monica s. lam, Ravi Sethi

SCoP: Region mit zusätzlichen Eigenschaften (wegen Polyedermodell):
Statisch berechenbar:
    Kontrollbedingungen
    Schleifengrenzen
ScopDetection
ScopInfo:
    Kann wieder abgelehnt werden (Kostenmodell, Annahmen scheitern doch, Annahmen treffen bei Modellierung (.Z.B. nie erfüllbar), Annahme z.B.: kein aliasing)
    Beispiel Regiontree

Versuchsaufbau:
    Wie weit ist das von Polly abgedeckte Spektrum ausreichend, um die interessanten SCoPs zu transformieren. Teil der Laufzeit abdecken. Wie hoch ist der Speedup? Anteile sequentieller Anteil. Amdals law (Anteil), Anteil der Laufzeit an Gesamtausführung.
    Stellenauswahl? Wie funktioniert das Profiling an sich? Empirische Arbeit. (Standardarbeit Medizin) Genaue Definition Kontrollvariablen (?) Experiment Setup and Execution; Metriken?, Variablen?, Konfiguration? (Messen, was Polly kann, messen, was die nächst größere kann) Welche interessant, welche nicht?, Hypothesen? (Erwartung?), Testprogramme (Versionen?, Programme?, benchbuild-version?, Integriert in benchbuild.) Variablen: Type, scale, Einheit, Bereich?) Mögliche Gründe für Abweichungen? (Wie geht man damit um?)
    (2013 Potential of Polyhedral model, Arbeit über Struktur empirischer Arbeiten)
    Domäne (Verbesserung innerhalb der Domäne?) Zitat related work
    Laufzeitanteil der nächst größeren Region?
    Interessanter noch: lmt (?)


Evaluierung:
Wo ist Luft nach oben?
Wie komme ich an die größere Region? Ablauf Instrumentierung? (Graphische Darstellung)
compilezeit-Option: Erweitert oder nicht.
Vergleich: Wie ändert sich die Laufzeit, wenn die Regionerweitert wird?



minted
\end{comment}

\chapter{Experimentelle Studie}
\section{Vorgehen}
\section{Resultate}
\section{Analyse}


\chapter{Quellen}
\nocite{*}
\printbibliography[heading=bibintoc, title={Literatur}, keyword=Literature]
\printbibliography[title={Diagramme}, title={Anhang}, keyword=Appendix]
\printbibliography[title={Sonstige}, title={Sonstige}, notkeyword=Literature, notkeyword=Diagram]

\end{document}
