\chapter{Experiment Setup And Execution}
\section{Goals}

\section{Measurement Methology}
\draftnote{What is measured?
    \begin{itemize}
        \item ratio sequential and parallel instructions
        \item reasons for parents of SCoPs being invalid as SCoP
        \item Theoretical impact if certain reasons would not appear
    \end{itemize}
}
\subsection{Definition ratio sequential/parallel}
Let \(S\in\mathbb{N}\) be the number of sequential instructions, \(P\in\mathbb{N}\) the number of instructions within a \scop and \(I\in\mathbb{N}\) the number of all instructions such that \(S + P = I\).
Then the sequential ratio is defined as:
\[R_{seq} := \frac{S}{I}\]
Analogue the \scop ratio is defined as:
\[R_{par} := \frac{P}{I}\]
\subsection[Amdahl's Law]{Amdahl's Law \cite{AmdahlsLaw}}
Let \(N\in\mathbb{N}\) be the number of processors and \(R_{seq}\in\mathbb{Q}\) be the sequential ratio.
Then the speedup \(S\in\mathbb{Q}\) is defined as:
\[S := \frac{N}{(R_{seq}*N)+(1-R_{seq})}\]
\subsection{Reasons for SCoPs being invalid}
These are the implemented criteria of Polly for rejecting a parent of a \scop being also a \scop.
\begin{itemize}
    \item Parent is toplevel (\autoref{lst:parentIsToplevel})
    \item Unsupported terminator instruction
    \item Unreachable in exit block (\autoref{lst:unreachableExitBlock})
    \item Irreducible loops (\autoref{lst:irreducibleLoops})
    \item Undefinded branch condition
    \item Non-integer branch condition
    \item Undefined operands in comparison
    \item Non-affine branch condition
    \item No base pointer
    \item Undefinded base pointer
    \item Variant base pointer (\autoref{lst:variantBasePointer})
    \item Non-affine memory accesses (\autoref{lst:nonAffineMemoryAccesses})
    \item Accesses with differing sizes
    \item Uncomputable loop bounds (\autoref{lst:uncomputeableLoopBounds})
    \item Loop without exit (\autoref{lst:loopWithoutExit})
    \item Function call with side effects (\autoref{lst:functionCallSideEffects})
    \item Complicated access semantics (volatile or atomic)
    \item Base address aliasing (\autoref{lst:baseAddressAliasing})
    \item Integer to pointer conversion (\autoref{lst:integerToPointer})
    \item Stack allocations
    \item Unknown instructions
    \item Contains entry block
    \item Assumed to be unprofitable (\autoref{lst:assumedUnprofitable})
\end{itemize}

\begin{comment}
    Following is copy\&pasted from pollys ScopDetectionDiagnostic.cpp

SCOP_STAT(CFG, ""),
SCOP_STAT(InvalidTerminator, "Unsupported terminator instruction"),
SCOP_STAT(UnreachableInExit, "Unreachable in exit block"),
SCOP_STAT(IrreducibleRegion, "Irreducible loops"),
SCOP_STAT(LastCFG, ""),
SCOP_STAT(AffFunc, ""),
SCOP_STAT(UndefCond, "Undefined branch condition"),
SCOP_STAT(InvalidCond, "Non-integer branch condition"),
SCOP_STAT(UndefOperand, "Undefined operands in comparison"),
SCOP_STAT(NonAffBranch, "Non-affine branch condition"),
SCOP_STAT(NoBasePtr, "No base pointer"),
SCOP_STAT(UndefBasePtr, "Undefined base pointer"),
SCOP_STAT(VariantBasePtr, "Variant base pointer"),
SCOP_STAT(NonAffineAccess, "Non-affine memory accesses"),
SCOP_STAT(DifferentElementSize, "Accesses with differing sizes"),
SCOP_STAT(LastAffFunc, ""),
SCOP_STAT(LoopBound, "Uncomputable loop bounds"),
SCOP_STAT(LoopHasNoExit, "Loop without exit"),
SCOP_STAT(FuncCall, "Function call with side effects"),
SCOP_STAT(NonSimpleMemoryAccess,
          "Compilated access semantics (volatile or atomic)"),
SCOP_STAT(Alias, "Base address aliasing"),
SCOP_STAT(Other, ""),
SCOP_STAT(IntToPtr, "Integer to pointer conversions"),
SCOP_STAT(Alloca, "Stack allocations"),
SCOP_STAT(UnknownInst, "Unknown Instructions"),
SCOP_STAT(Entry, "Contains entry block"),
SCOP_STAT(Unprofitable, "Assumed to be unprofitable"),
SCOP_STAT(LastOther, "")
\end{comment}

\section{Experiment Variables}
\begin{table}[H]
    \myfloatalign
    \begin{tabularx}{\textwidth}{Xccccc} \toprule
        \tableheadline{Name} & \tableheadline{Abbr.} & \tableheadline{Type} & \tableheadline{Scale Type} & \tableheadline{Unit} & \tableheadline{Range} \\ \midrule
        Ratio sequential & \(R_{seq}\) & ? & Ratio & \% & \([0, 100]\)\\
        Ratio \scop & \(R_{par}\) & ? & Ratio & \% & \([0, 100]\)\\
        Speedup & \(S\) & ? & Ratio & ? & \(\mathbb{Q}\)\\
        \bottomrule
    \end{tabularx}
    \caption{Experiment Variables}
\end{table}

\section{Hypotheses}

\section{Subject Programs}
\begin{table}[H]
    \myfloatalign
    \begin{tabularx}{\textwidth}{XcX} \toprule
        \tableheadline{Name} & \tableheadline{Version} & \tableheadline{Tested Inputs} \\ \midrule
        benchbuild\\
        llvm\\
        clang\\
        polly\\
        polli\\ \midrule
        7z\\
        Rasdam\\
        bzip2\\
        ccrypt\\
        crafty\\
        crocopat\\
        ffmpeg\\
        gzip\\
        js\\
        lammps\\
        lapack\\
        leveldb\\
        libressl\\
        linpack\\
        lulesh\\
        lulesh-omp\\
        mcrypt\\
        minisat\\
        openblas\\
        postgres\\
        povray\\
        python\\
        ruby\\
        sqlite3\\
        tcc\\
        x264\\
        xz\\
        \bottomrule
    \end{tabularx}
    \caption[Subject programs]{Subject programs and benchbuild used. (Versions in parenthesis represent git hashes)}
\end{table}

\section{Tasks}
\draftnote{
    \begin{enumerate}
        \item Create Pass (\autoref{subsec:optimizer})
        \begin{itemize}
            \item How does the pass work?
        \end{itemize}
        \item Run benchbuild over the group \enquote{benchbuild}
        \item Analyse output
            \begin{itemize}
                \item Most common reasons for rejection?
                \item Ratio of SCoPs?
                \item Possible effect when theoretically solve reasons for rejection?
            \end{itemize}
    \end{enumerate}
}

\section{Design}

\section{Experiment Setting}

\section{Deviations}

\section{Analyse}
