\chapter{Analysis}
\section{Descriptive Statistics}
\subsection{Ratios and Speedups}
\begin{figure}[!h]
    \caption[Comparison of the distribution of \(DynCov_s\) and \(DynCov_p\)]{
        Here the distribution of \(DynCov_s\) and \(DynCov_p\) are visualized.
        The broader the plot is the more often the value appeared over the projects.
    }
    \includesvg[width=\textwidth]{compDyncovScopParent}
\end{figure}
The measurements taken are listed in \autoref{tab:ratiosAndSpeedups}.
\begin{table}[!h]
    \resizebox{!}{\textwidth}{
        \begin{minipage}{\textwidth}
            \myfloatalign
            \begin{tabularx}{\textwidth}{Xcccc} \toprule
        \tableheadline{Project} & \tableheadline{\(R_s\)} & \tableheadline{\(R_p\)} & \tableheadline{\(S_s\)} & \tableheadline{\(S_p\)}\\\midrule
                \textbf{Benchmarks}\\
                MultiSourceApplications & X & X & X & X\\
                MultiSourceBenchmarks & X & X & X & X\\
                SingleSourceBenchmarks & X & X & X & X\\
                SPEC2006 & X & X & X & X\\
                Povray & X & X & X & X\\
                \midrule
                \textbf{Compression}\\
                7z & X & X & X & X\\
                bzip2 & X & X & X & X\\
                gzip & X & X & X & X\\
                xz & X & X & X & X\\
                \midrule
                \textbf{Database}\\
                Leveldb & X & X & X & X\\
                SQLite3 & X & X & X & X\\
                \midrule
                \textbf{Encryption}\\
                ccrypt & X & X & X & X\\
                libressl & X & X & X & X\\
                \midrule
                \textbf{Multimedia}\\
                ffmpeg & X & X & X & X\\
                povRAY & X & X & X & X\\
                x264 & X & X & X & X\\
                \midrule
                \textbf{Scientific}\\
                LAMMPS & X & X & X & X\\
                LAPACK & X & X & X & X\\
                LINPACK & X & X & X & X\\
                \midrule
                \textbf{Simulation}\\
                crafty & X & X & X & X\\
                lulesh & X & X & X & X\\
                lulesh-omp & X & X & X & X\\

                \midrule
                \textbf{Verification}\\
                minisat & X & X & X & X\\
                \midrule
                \textbf{Other}\\
                js & X & X & X & X\\
                openblas & X & X & X & X\\
                Rasdaman & X & X & X & X\\
                \bottomrule
            \end{tabularx}
            \caption[Measured ratios and speedups]{
                Measured ratios and speedups.
                This table lists the ratio of the \scops -- this means the parts which can be optimized by Polly -- according to the execution time of the tests of the project.
                \draftnote{A \enquote{X} indicates that no data is available either because the run time tests of the project or the measured project itself is broken.}
            }
            \label{tab:ratiosAndSpeedups}
        \end{minipage}
    }
\end{table}
\subsection{Reasons for rejecting parents}
\autoref{fig:rejectReasonsGrouped} indicates that the most common reason for parents of \scops being rejected is that it is already the toplevel region.
\piechart{
      46.6/Chrome,
      24.6/Internet Explorer,
      20.4/Firefox,
      5.1/Safari,
      1.3/Opera,
      2.0/Other
}{Reasons for parents being rejected}{fig:rejectReasonsGrouped}
But looking at the low coverage of the \scops these parents have to be toplevel regions of quiet small functions.\\
Another common reason for rejection is that loop bounds can not be computed at compile time.
This happens \eg when the boundaries of a loop are depending on a non-const variable like a parameter.
Such a loop can be found in lammps (\autoref{lst:couldNotCompute}).
\begin{code}
    \caption{An example for 'loop bound could not be computed'}
    \inputminted{c}{c/nonAffineLoopBoundCouldNotCompute.c}
    \label{lst:couldNotCompute}
\end{code}
The third big block of parents is rejected because they contain call instructions like the calls to \texttt{rand(...)} in \autoref{lst:matmulcpp}.
These can not be optimized due to not being able to guarantee the absence of side effects.\\
\draftnote{In \autoref{tab:rejectionReasons} the complete data can be found.}
\section{Hypothesis Testing}
\(H_1\): Observing the given programs and test sets (\autoref{tab:subjectPrograms}) the most common reason for rejecting parents of \scop to be valid as well is truly that the parent already is the toplevel region.\\
\(H_2\): The ratio both \(R_s\) and \(R_p\) are rather low although \(R_p\) is significally higher than \(R_s\).
The reason for it is that there are many regions classified by Polly as \enquote{unprofitable} which have parents far bigger than them self.\\
\(H_3\): \draftnote{As result the speedups \(S_s\) and \(S_p\) are not as high as expected even considering that \(S_p\) is higher than \(S_s\).}\\
