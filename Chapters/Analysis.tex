\chapter{Analysis}
All results of the measurements are listed in \autoref{sec:dataTables}.
The complete set of all data is available at \draftnote{Is it going to?}.

\section{Descriptive Statistics}
In \autoref{fig:coverages} the distribution density of \dyncovs (left) and \dyncovp (right) over the projects is shown.
The plots also contain marks for 25\% quantiles.
The higher these lines the more projects have a higher coverage observing the regions automatically optimizable by Polly.
\begin{figure}[!h]
    \caption[Distribution density of \dyncovs and \dyncovp]{
        Distrubution density of \dyncovs and \dyncovp as violin plots.
        Each plot is divided in four quantiles with same size.
    }
    \includesvg[width=\textwidth]{compDyncovScopParent}
    \label{fig:coverages}
\end{figure}
In both plots the lowest quantile is the broadest one and has almost the same size.
Also in both plots the quantile between 25\% and 50\% is the thinnest one.
When looking at the left plot the quantile between 50\% and 75\% is broader than the upper one.
In the right plot it is the other way round.\\
As a result the speedup \(S_p=\speedupMaxRegions\) is only slightly higher than \(S_s=\speedupScops\).\\
\autoref{tab:statsMatrix} shows the mean (\(\mu\)), the standard deviation (s) and the variance (\(s^2\)) of either \dyncovp and \dyncovs and how much \dyncovp is bigger than \dyncovs (\(\Delta\)) in \%.
\begin{table}[!h]
    \myfloatalign
    \begin{tabularx}{\textwidth}{lCCC}
        \tableheadline{} & \tableheadline{SCoPs} & \tableheadline{\(\Delta\)} & \tableheadline{MaxRegions}\\\toprule
        \csvreader[head to column names]{csv/statsMatrix.csv}{}{\csvcoli&\csvcolii&\csvcoliii&\csvcoliv\\}
        \\\bottomrule
    \end{tabularx}
    \caption[Statistical evaluations of \dyncovp and \dyncovs]{
        This table contains the mean (\(\mu\)), standard deviation (s), the variance (\(s^2\)) of either \dyncovp and \dyncovs and how much \dyncovp is bigger than \dyncovs (\(\Delta\)) in \%.
    }
    \label{tab:statsMatrix}
\end{table}
The mean value is almost the same either looking at the \scops or the MaxRegions.
The variance and the standard deviation of \dyncovp are higher than of \dyncovs.\\
\autoref{fig:rejectReasonsGrouped} indicates that the most common reason for parents of \scops being rejected are that it is already the toplevel region, the loop bound could not be computed and calls to functions with (possible) side effects.
\begin{figure}[!h]
    \caption[Reasons for rejecting SCoPs]{
        Here the most common reasons for rejecting are listed.
        The larger the tile the more often it appears as reason for rejecting a parent as \scop.
        But a large reason does not automatically imply that it has also a big impact on the run time.
    }
    \includesvg[width=\textwidth]{pieInvalidReasons}
    \label{fig:rejectReasonsGrouped}
\end{figure}
The concrete values can be found in \autoref{tab:rejectionReasons}.

\section{Hypothesis Testing}
To check the significance of the collected data first the independence of the two datasets \scops and MaxRegions has to be stated.
Further a Shapiro-Wilk-Test \cite{shapiroWilkTest} reveals that neither of these datasets is distributed normally.
All p values are listed in \autoref{tab:pValues}.\\
\begin{table}[!h]
    \myfloatalign
    \begin{tabularx}{\textwidth}{Xcc}
        \tableheadline{Test} & \tableheadline{SCoPs} & \tableheadline{MaxRegions}\\\toprule
        \csvreader[head to column names]{csv/pValues.csv}{}{\csvcoli&\csvcolii&\csvcoliii\\}
        \\\bottomrule
    \end{tabularx}
    \caption[P values of tests]{This table contains all p values neccessary to be caculated for ensuring the correctness of the hyptheses.}
    \label{tab:pValues}
\end{table}
\(H_1\): Observing the given programs and test sets (\autoref{tab:subjectPrograms}) the most common reason for rejecting parents of \scops to be valid as well is truly that the parent already is the toplevel region.\\
\(H_2\): The mean of \dyncovs is \draftnote{significally?} lower than \hTwoAbout.
Even \dyncovp is as well\drafnote{?}.\\
\(H_3\): A Mann-Whitney-U-Test \cite{utest} reveals that \dyncovp is not significally higher than \dyncovs (p value: \utestPValue).\\
\draftnote{
    \(R_1\): All parents rejected for being a toplevel region can only eliminated by integrating them into a larger context -- this means inlining.\\
    The regions rejected because Polly is not able to determine loop bounds may be eliminated at least partially when including run time information using \jit compilation.
}
