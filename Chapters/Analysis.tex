\chapter{Analysis}
All results of the measurements are listed in \autoref{sec:dataTables}.
The complete set of all data is available at \drafnote{Is it going to?}.

\section{Descriptive Statistics}
\subsection{Coverages}
In \autoref{fig:coverages} the distribution of \dyncovs (left) and \dyncovp (right) over the projects is shown.
There are three lines marking where 25\%, 50\% and 75\% of the projects are located.
\begin{figure}[!h]
    \caption[Comparison of the distribution of \dyncovs and \dyncovp]{
        Here the distribution of \dyncovs and \dyncovp are visualized.
        The broader the plot is the more often the value appeared over the projects.
    }
    \includesvg[width=\textwidth]{compDyncovScopParent}
    \label{fig:coverages}
\end{figure}
Looking at \autoref{fig:coverages} the coverage of the lower 25\% of the project did not change visibly even when extending the \scops to the size of their parents if any.
Although the lower half and the upper quarter of the project show increasments of the coverage.\\
\autoref{tab:statsMatrix} shows the mean (\(\mu\)), the standard deviation (s) and the variance (\(s^2\)) of either \dyncovp and \dyncovs and the difference between (\(\Delta\)) them (in \%).
\begin{table}[!h]
    \myfloatalign
    \begin{tabularx}{.5\textwidth}{lc}
        \tableheadline{SCoPs} & \tableheadline{MaxRegions}\\\toprule
        \csvreader[head to column names]{csv/statsMatrix.csv}{}{\csvcoli&\csvcolii\\}
        \\\bottomrule
    \end{tabularx}
    \caption[Statistical evaluations of \dyncovp and \dyncovs]{
        This table contains the mean (\(\mu\)), standard deviation (s), the variance (\(s^2\)) of either \dyncovp and \dyncovs and the difference between (\(\Delta\)) them (in \%).
    }
    \label{tab:statsMatrix}
\end{table}
\subsection{Reasons for rejecting parents}
\autoref{fig:rejectReasonsGrouped} indicates that the most common reason for parents of \scops being rejected is that it is already the toplevel region.
\begin{figure}[!h]
    \caption[Reasons for rejecting SCoPs]{
        Here the most common reasons for rejecting are listed.
        The larger the tile the more often it appears as reason for rejecting a parent as \scop.
        But a large reason does not automatically imply that it has also a big impact on the run time.
    }
    \includesvg[width=\textwidth]{pieInvalidReasons}
    \label{fig:rejectReasonsGrouped}
\end{figure}
%\piechart{
%      46.6/Chrome,
%      24.6/Internet Explorer,
%      20.4/Firefox,
%      5.1/Safari,
%      1.3/Opera,
%      2.0/Other
%}{Reasons for parents being rejected}{fig:rejectReasonsGrouped}
But looking at the low coverage of the \scops these parents have to be toplevel regions of quiet small functions.\\
Another common reason for rejection is that loop bounds can not be computed at compile time.
This happens \eg when the boundaries of a loop are depending on a non-const variable like a parameter.
Such a loop can be found in lammps (\autoref{lst:couldNotCompute}).
\begin{code}
    \caption{An example for 'loop bound could not be computed'}
    \inputminted{c}{c/nonAffineLoopBoundCouldNotCompute.c}
    \label{lst:couldNotCompute}
\end{code}
The third big block of parents is rejected because they contain call instructions like the calls to \texttt{rand(...)} in \autoref{lst:matmulcpp}.
These can not be optimized due to not being able to guarantee the absence of side effects.\\
\draftnote{In \autoref{tab:rejectionReasons} the complete data can be found.}

\section{Hypothesis Testing}
\(H_1\): Observing the given programs and test sets (\autoref{tab:subjectPrograms}) the most common reason for rejecting parents of \scop to be valid as well is truly that the parent already is the toplevel region.\\
\(H_2\): The ratio both \(R_s\) and \(R_p\) are rather low although \(R_p\) is significally higher than \(R_s\).
The reason for it is that there are many regions classified by Polly as \enquote{unprofitable} which have parents far bigger than them self.\\
\(H_3\): \draftnote{As result the speedups \(S_s\) and \(S_p\) are not as high as expected even considering that \(S_p\) is higher than \(S_s\).}\\
