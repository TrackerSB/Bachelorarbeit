%********************************************************************
% Appendix
%*******************************************************
% If problems with the headers: get headings in appendix etc. right
%\markboth{\spacedlowsmallcaps{Appendix}}{\spacedlowsmallcaps{Appendix}}
\begin{landscape}
\chapter{Appendix}
This chapter contains mainly lots of the source code and complete tables of data referenced within the study.
The main purpose is for getting an even more detailed impression of the used or generated files and for being able accurate reproducing the results of the study.
%\graffito{More dummy text.}

\section{Code Listings}
\subsection{Matmul.cpp Code Listings}
The following listings contain the (generated) code of the example matrix multiplication which is often used throughout the background chapters.
\begin{code}
    \caption{LLVM-IR of \autoref{lst:matmulcpp}}
    \inputminted{LLVM}{ll/matmul.ll}
    \label{lst:matmulll}
\end{code}
\begin{code}
    \caption{LLVM-IR (O3 optimized) of \autoref{lst:matmulcpp}}
    \inputminted{LLVM}{ll/matmulO3.ll}
    \label{lst:matmulllO3}
\end{code}
\begin{code}
    \caption{LLVM-IR of \autoref{lst:matmulcpp} prepared for Polly}
    \inputminted{LLVM}{ll/matmul.preopt.ll}
    \label{lst:matmulpreoptll}
\end{code}
\begin{code}
    \caption[Program for checking overhead]{The program used for checking the overhead of the measurement itself}
    \inputminted{c++}{cpp/checkMeasurementOverhead.cpp}
    \label{lst:checkOverhead}
\end{code}
\begin{code}
    \caption[Script for plots and statistic]{This R script contains all commands for filtering the data, generating the plots, calculating statistical values and p values and annotating the projects of \lnt.}
    \inputminted{R}{r/generatePlotsAndTables.R}
    \label{lst:rscript}
\end{code}
\end{landscape}

\subsection{Examples for invalid SCoPs}
These peaces of code illustrate simple situations where \scops are invalid.
\begin{code}
    \caption{Parent is top level}
    \inputminted{c++}{cpp/ParentIsTopLevelRegion.cpp}
    \label{lst:parentIsToplevel}
\end{code}
\begin{code}
    \caption{Unreachable in exit block}
    \inputminted{c++}{cpp/UnreachableInExitBlock.cpp}
    \label{lst:unreachableExitBlock}
\end{code}
\begin{code}
    \caption{Irreducible loops}
    \inputminted{c++}{cpp/IrreducibleRegion.cpp}
    \label{lst:irreducibleLoops}
\end{code}
\begin{code}
    \caption{Variant base pointer}
    \inputminted{c++}{cpp/VariantBasePointer.cpp}
    \label{lst:variantBasePointer}
\end{code}
\begin{code}
    \caption{Non-affine memory accesses}
    \inputminted{c++}{cpp/NonAffineMemoryAccesses.cpp}
    \label{lst:nonAffineMemoryAccesses}
\end{code}
\begin{code}
    \caption{Uncomputable loop bounds}
    \inputminted{c++}{cpp/UncomputableLoopBounds.cpp}
    \label{lst:uncomputeableLoopBounds}
\end{code}
\begin{code}
    \caption{Loop without exit}
    \inputminted{c++}{cpp/LoopWithoutExit.cpp}
    \label{lst:loopWithoutExit}
\end{code}
\begin{code}
    \caption{Function call with side effects}
    \inputminted{c++}{cpp/FunctionCall.cpp}
    \label{lst:functionCallSideEffects}
\end{code}
\begin{code}
    \caption{Base address aliasing}
    \inputminted{c++}{cpp/BaseAddressAliasing.cpp}
    \label{lst:baseAddressAliasing}
\end{code}
\begin{code}
    \caption{Integer to pointer conversion}
    \inputminted{c++}{cpp/IntegerToPointerConversion.cpp}
    \label{lst:integerToPointer}
\end{code}
\begin{code}
    \caption{Assumed to be unprofitable}
    \inputminted{c++}{cpp/AssumedToBeUnprofitable.cpp}
    \label{lst:assumedUnprofitable}
\end{code}

\section{Specifications of machine used for measurements}
\begin{code}
    \captionsetup{type=table}
    \inputminted{text}{gfx/zeusLscpu.log}
    \caption[Specifications of the CPUs used for measurement]{
        Specifications of the CPUs used for measurement.
        More precisely it is simply the output of \texttt{lscpu}.
    }
    \label{tab:lscpu}
\end{code}
\begin{code}
    \captionsetup{type=table}
    \inputminted{text}{gfx/zeusMeminfo.log}
    \caption[Specifications of the memory of the machine used for measurement]{
        Specifications of the memory of the machine used for measurement.
        More precisely this is the output of \texttt{cat /proc/meminfo}.
    }
    \label{tab:meminfo}
\end{code}

\begin{landscape}
\section{Data Tables}\label{sec:dataTables}
Here is the hole data listed collected throughout the entire study.
\LTXtable{\textheight}{tables/ratiosScops.tex}
\LTXtable{\textheight}{tables/ratiosMaxRegions.tex}
\LTXtable{\textheight}{tables/invalidReasons.tex}
\end{landscape}

\begin{comment}
    \section{Appendix Section Test}
    Test: \autoref{tab:moreexample} (This reference should have a
    lowercase, small caps \spacedlowsmallcaps{A} if the option
    \texttt{floatperchapter} is activated, just as in the table itself
     $\rightarrow$ however, this does not work at the moment.)

    \begin{table}[h]
        \myfloatalign
      \begin{tabularx}{\textwidth}{Xll} \toprule
        \tableheadline{labitur bonorum pri no} & \tableheadline{que vista}
        & \tableheadline{human} \\ \midrule
        fastidii ea ius & germano &  demonstratea \\
        suscipit instructior & titulo & personas \\
        %postulant quo & westeuropee & sanctificatec \\
        \midrule
        quaestio philosophia & facto & demonstrated \\
        %autem vulputate ex & parola & romanic \\
        %usu mucius iisque & studio & sanctificatef \\
        \bottomrule
      \end{tabularx}
      \caption[Autem usu id]{Autem usu id.}
      \label{tab:moreexample}
    \end{table}

    \section{Another Appendix Section Test}
    Equidem detraxit cu nam, vix eu delenit periculis. Eos ut vero
    constituto, no vidit propriae complectitur sea. Diceret nonummy in
    has, no qui eligendi recteque consetetur. Mel eu dictas suscipiantur,
    et sed placerat oporteat. At ipsum electram mei, ad aeque atomorum
    mea. There is also a useless Pascal listing below: \autoref{lst:useless}.

    \begin{lstlisting}[float=b,language=Pascal,frame=tb,caption={A floating example (\texttt{listings} manual)},label=lst:useless]
    for i:=maxint downto 0 do
    begin
    { do nothing }
    end;
    \end{lstlisting}
\end{comment}
