\chapter{Related Work}
There are also other studies investigating other aspects of Polly, the polyhedral or alternative technologies.

\section{Alternative Studies}
Andreas Simbürger et. al. \cite{PolyhedralEmpiricalStudy} investigated the potential of polyhedral compilation in an empirical study.
He also compares the coverages of dynamic and static analysis and concludes that the polyhedral model \enquote{is a well-studied and promising approach to automatic program optimization} \cite{PolyhedralEmpiricalStudy}.
But also mentions \enquote{that the current practical implementations of the polyhedral model do not achieve practically relevant execution coverages, when applied to real-world programs at compile time} \cite{PolyhedralEmpiricalStudy}.\\
In contrast to this study it is neither taking care about the reasons for rejecting parents of \scops nor does it investigate the potential of extending them.

\section{Alternative Extensions}\label{sec:altExt}
There are other studies which already investigated certain aspects of rejection reasons and how to possibly solve some of them.\\
In \cite{spolly} also the rejection reasons are investigated and the impact of speculative extensions to Polly.
In this paper is determined that many of the rejection reasons are a result from the \enquote{conservative overapproximation of the employed static analysis} \cite{spolly}.
So the approach of SPolly was done which integrates run time knowledge and features for replacing the overapproximation.
According the the paper SPolly is able to \enquote{effectively widen the applicability of polyhedral optimization} \cite{spolly}.

\section[Alternative Technologies]{Alternative Technologies \cite{PolyhedralEmpiricalStudy}}
The \llvm is not the only framework that can handle the polyhedral model.\\
There are other systems which extract \scops directly from the source code.
The first one was LooPo \cite{loopo}.
A current system is PoCC \cite{pocc} which implements a full compiler tool chain for applying automatically optimizations based on the polyhedral model.
It supports two tools for transformations.
This one is PLuTo \cite{pluto} and the other is LaTSeE \cite{latsee}.
In \cite{PolyhedralEmpiricalStudy} both are discribed as follows:
\begin{quotation}\noindent
    \enquote{The PLuTo scheduling algorithm implements a transformation that optimizes data locality on shared-memory systems.
    Rather than generating the optimal solution, LeTSeE tries to converge on it iteratively by exploring the legal transformation space.}
\end{quotation}
