\chapter{LLVM}
\section{Geschichte \cite{LLVMResearchBeginning}}
Die \ac{LLVM} ist eine Zusammenfassung von modularen und wiederverwendbaren Compilern und Toolchain Technologien zur Unterstützung von sowohl statischer als auch dynamischer Kompilation sowie transparenter und ?lifelong? Programm Analysen und Transformationen für beliebige Programme. \cite{LLVMWebsite}\\
Die \ac{LLVM} war ursprünglich ein Forschungsprojekt an der Universität in Illinois, das 2004 erstmals in einer Veröffentlichung erläutert wurde TODO: Ändere Formulierung.
Aufgrund des starken Zuwachses an Community und der steigenden Zahl an Unterprojekten wurde 2014 die \enquote{\ac{LLVM} Foundation} gegründet, welche seitdem das Projekt verwaltet. \cite{LLVMFoundation}\\
Um die Optimierungen und Analysen von der ursprünglich verwendeten Programmiersprache möglichst unabhängig zu machen, wurde die \ac{LLVM-IR} entworfen.

\section{Pipeline \cite{IntroLLVM}}
Die Pipeline der \ac{LLVM} (\autoref{fig:llvmPipeline}) erhält als Eingabe beliebigen Quelltext, welcher in die \ac{LLVM-IR} übersetzt wird, auf der die folgenden Komponenten arbeiten.
Am Ende der Pipeline wird die \ac{LLVM-IR} in ein ausführbares Binary übersetzt.
\begin{figure}[!ht]
    \caption{Die Pipeline der \ac{LLVM}}
    \label{fig:llvmPipeline}
    \centering
    \begin{tikzpicture}
        \node(languages)[nonLlvmIrNode]{C/C++/Obj-C};
        \node(clang)[nonLlvmIrNode, right=of languages]{clang frontend};
        \node(opt)[llvmIrNode, right=of clang]{\ac{LLVM} Optimizer};
        \node(generator)[llvmIrNode, below=of opt]{\ac{LLVM} Code Generator};
        \node(linker)[nonLlvmIrNode, left=of generator]{\ac{LLVM} Linker};
        \path[nonLlvmIrPath] (languages) to (clang);
        \path[llvmIrPath] (clang) to (opt);
        \path[llvmIrPath] (opt) -| ($(opt.north east) + (0.5,0.5)$) node[auto]{Pass} -| (opt);
        \path[llvmIrPath] (opt) to (generator);
        \path[nonLlvmIrPath] (generator) to (linker);
    \end{tikzpicture}
\end{figure}\\
\subsection{clang frontend}
Genauer wird beliebiger Quelltext -- zurzeit werden vor allem C, C++ und C-nahe Sprachen unterstützt -- mittels dem clang frontend und der Option \texttt{-emit-llvm} in die \ac{LLVM-IR} übersetzt.
Wird zusätzlich das Flag \texttt{-S} gesetzt, erzeugt clang eine Textdatei mit \ac{LLVM-IR}; ohne diese Option wird ein \ac{LLVM-IR}-Binary erzeugt.
\subsection{LLVM Optimizer}\label{subsec:optimizer}
Auf dieser Zwischensprache können durch den \ac{LLVM} Optimizer über den Befehl opt beliebige Analysen und Optimierungen durchgeführt werden.
Die Schritte, die dabei von opt durchgeführt und per Flag aktiviert werden können, werden als \enquote{Passes} bezeichnet.
\subsection{LLVM Code Generator}
Wurden die gewünschten Passes durchgeführt, übersetzt der \ac{LLVM} Code Generator die \ac{LLVM-IR} in Assembler.
\subsection{LLVM Linker}
Der \ac{LLVM} Linker kompiliert daraus ein ausführbares Binary.
