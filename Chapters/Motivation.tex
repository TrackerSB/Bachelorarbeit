\chapter{Motivation}
\draftnote{
    As the number of cores of processors, the variety of devices and platforms is increasing there is a need for an efficient compiler infrastructure which is capable of performing language and platform independent optimizations for being able to utilize the capacity of the underlying hardware.
    So projects like LLVM and Polly were developed for dealing with it.
    But there is the need of checking whether the impact of the optimzations could be bigger referencing the sizes of the areas automatically optimizable by Polly.
}

\draftnote{
    \begin{itemize}
        \item problem statement: (What is the problem?, Where does it occur?, Who has observed it?, Why is it important to be solved?, brief solution idea, expected benefits)
        \item research object: Goal/Question/Metric (GQM) (Analyze <Object(s) of study> for the purpose of <purpose> with respect to their <Quality Focus> from the point of view of the <Perspective> in the context of <context>.)
        \item context: application type, application domain, type of company, experience of participants, time constraints, process, tools, size of project, specific requirements with regard to the support of the environment.
    \end{itemize}
}
