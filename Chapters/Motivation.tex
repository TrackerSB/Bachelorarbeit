\chapter{Motivation}
As the number of cores of processors and the variety of devices and platforms is increasing there is a need for an efficient compiler infrastructure which is capable of performing language and platform independent optimizations allowing to utilize the capacity of the underlying hardware as much as possible.
A popular approach of such a compiler infrastructure is LLVM.
This infrastructure has options to extend it by adding plugins like Polly.
Polly adds the ability to apply further optimizations on the basis of the polyhedral model.
When using Polly there arises the question about the ratio of the parts which can currently be automatically optimized by Polly, what are the common reasons of these parts for not being even bigger and the theoretical speedup.\\
Answering this question gives on the one hand an assessement to whether Polly is already able to optimize a high percentage of the instructions of given programs -- commonly used and installed -- and on the other hand a potential to investigate the main reasons for the automatically optimizable parts not being as big as expected and the theoretical impact when being able eliminating specific reasons.

\draftnote{
    \begin{itemize}
        \item problem statement: (What is the problem?, Where does it occur?, Who has observed it?, Why is it important to be solved?, brief solution idea, expected benefits)
        \item research object: Goal/Question/Metric (GQM) (Analyze <Object(s) of study> for the purpose of <purpose> with respect to their <Quality Focus> from the point of view of the <Perspective> in the context of <context>.)
        \item context: application type, application domain, type of company, experience of participants, time constraints, process, tools, size of project, specific requirements with regard to the support of the environment.
    \end{itemize}
}
