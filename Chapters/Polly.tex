\chapter{Polly}
Polly\footnote{Der Name Polly ist eine Kombination von \enquote{Polyhedral} und \enquote{\ac{LLVM}} \cite{PollyGrosser}} ist eines der zahlreichen Unterprojekte der \ac{LLVM} und stellt Möglichkeiten bereit, automatische Optimierungen auf Basis des Polyedermodells auf \ac{LLVM-IR} durchzuführen.
Diese Optimierungen können über Flags des \ac{LLVM} Optimizers genutzt werden.
\section{Geschichte}
Polly wurde mit der Veröffentlichung (\cite{PollyGrosser}) eingeführt. TODO Mehr Informationen?

\section{Pipeline \cite{PollyPresentation}}
Die Einbindung von Polly wird über den \ac{LLVM} Optimizer realisiert, indem die Bibliothek von Polly geladen wird (\autoref{subsec:optimizer}).
\begin{figure}[h]
    \caption{Die Pipeline von Polly}
    \centering
    \begin{tikzpicture}
        \coordinate(clang);
        \node(opt)[llvmIrNode, right=of clang]{\ac{LLVM} Optimizer};
        \node(polly)[llvmIrNode, below=of opt]{\ac{LLVM} Polly};
        \coordinate[right=of opt](generator);
        \path[llvmIrPath] (clang) to (opt);
        \path[llvmIrPath, bend right] (opt.south west) to node[auto, swap]{SCoP Detection} (polly);
        \path[llvmIrPath, bend right] (polly) to node[auto, swap]{Code Generation} (opt.south east);
        \path[llvmIrPath] (opt) to (generator);
        \path[llvmIrPath] (polly) edge[loop below] ();
    \end{tikzpicture}
\end{figure}\\
Zwei essentielle Begriffe von Polly sind \enquote{Region} und \enquote{\ac{SCoP}}.
\subsection{Definition Region}

\subsection{Definition SCoP}
