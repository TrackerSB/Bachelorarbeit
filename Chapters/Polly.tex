\chapter{Polly}
Polly ist eines der zahlreichen Unterprojekte der \ac{LLVM} und stellt Möglichkeiten bereit, automatische Optimierungen auf Basis des Polyedermodells auf \ac{LLVM-IR} durchzuführen.
Diese Optimierungen können über den \ac{LLVM} Optimizer genutzt werden.
\section{Geschichte}
Polly wurde mit der Veröffentlichung (\cite{PollyGrosser}) eingeführt. TODO Mehr Informationen?

\section{Pipeline \cite{PollyPresentation}}
Die Einbindung von Polly wird über den \ac{LLVM} Optimizer realisiert.
\begin{center}
    \begin{tikzpicture}
        \coordinate(clang);
        \node(opt)[myNode, right=of clang]{\ac{LLVM} Optimizer};
        \node(polly)[myNode, below=of opt]{\ac{LLVM} Polly};
        \coordinate[right=of opt](generator);
        \path[myPath] (clang) to node[auto]{\ac{LLVM-IR}}(opt);
        \path[myPath, bend right] (opt.south west) to (polly);
        \path[myPath, bend right] (polly) to (opt.south east);
        \path[myPath] (opt) to node[auto]{\ac{LLVM-IR}}(generator);
        \path[myPath] (polly) edge[loop below] ();
    \end{tikzpicture}
\end{center}
\subsection{Definition Region}

\subsection{Definition \ac{SCoP}}
